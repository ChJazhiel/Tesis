\documentclass[10pt]{beamer}
\usepackage{graphicx}
\usepackage{subfigure}
\usepackage{media9}
\usepackage{animate}
\usepackage{movie15}
\usepackage{multicol}
\usepackage{wrapfig}
\usepackage{lipsum}
\usepackage{hyperref}
\usepackage{color}
\usepackage{multirow} % para las tablas
\usetheme{Berlin}
%\usetheme[realshadow,corners=2pt,padding=2pt]{chamfered}
%\usecolortheme{shark}

\usepackage{tikz}
\newcommand<>{\hover}[1]{\uncover#2{%
 \begin{tikzpicture}[remember picture,overlay]%
 \draw[fill,opacity=0.4] (current page.south west)
 rectangle (current page.north east);
 \node at (current page.center) {#1};
 \end{tikzpicture}}
}

\title{Modelos de Materia Oscura: Una Perspectiva Num\'erica}
\author{Jazhiel Chac\'on Lavanderos}
\institute{Escuela Superior de F\'isica y Matem\'aticas, I.P.N.}
\date{\today}

\begin{document}

\begin{frame}
\maketitle
\end{frame}

%\begin{frame}
%\frametitle{Contenido}
%\begin{itemize}
%\setlength{1em}
%\item Introducci\'on \\
%\item Simulaciones de $N$-Cuerpos \\
%\item El C\'odigo  \\
%\item Resultados \\
%\item Conclusiones y Perspectivas
%\end{itemize}
%\end{frame}
\begin{frame}
 \frametitle{Contenido}
\begin{itemize}
  \setlength\itemsep{2em}
  \item Introducci\'on
  \item Simulaciones de $N$-cuerpos
  \item El C\'odigo
  \item Resultados
  \item Conclusiones
\end{itemize}
\end{frame}
%%%%%%%%%%%%%%%%%%%%%%%%%%%%%%%%%%%%%%%%%%%%%%%%%%%%%%%%%%%%%%%%%%%%%%%%%%%%%%%%%%%%%%%%%%%%%%%%%%%%%%%%%%%%%%%%%%%%%%%%
%\begin{frame}
% \frametitle{Resumen}
%La naturaleza de la materia oscura ha sido y sigue siendo desconocida, f\'isicos te\'oricos y experimentales han trabajado en conjunto para poder descubrir su composici\'on durante las \'ultimas d\'ecadas. Una de las alternativas es crear simulaciones basadas en teor\'ias ya establecidas para poder explicar la existencia de la materia oscura y su composici\'on.
%\end{frame}
%%%%%%%%%%%%%%%%%%%%%%%%%%%%%%%%%%%%%%%%%%%%%%%%%%%%%%%%%%%%%%%%%%%%%%%%%%%%%%%%%%%%%%%%%%%%%%%%%%%%%%%%%%%%%%%%%%%%%%%%
\begin{frame}
 \frametitle{Principio Cosmol\'ogico}
\begin{itemize}
 \setlength\itemsep{1em}
 \item A grandes escalas, el Universo es homog\'eneo e is\'otropo en un espacio tridimensional.
 \item \textbf{Homogeneidad}  $\rightarrow$ uniformemente distribuido (galaxias uniformemente distribuidas a gran escala).
 \item \textbf{Isotrop\'ia} $\rightarrow$ Sin importar la direcci\'on de observaci\'on, las propiedades en el Universo son las mismas.
\end{itemize} 
\begin{figure}[h]
  \includegraphics[width=0.4\textwidth]{./Figuras/WMAP}
  \includegraphics[width=0.35\textwidth, height=0.3\textwidth]{./Figuras/sdss}
\end{figure}
\end{frame}

%%%%%%%%%%%%%%%%%%%%%%%%%%%%%%%%%%%%%%%%%%%%%%%%%%%%%%%%%%%%%%%%%%%%%%%%%%%%%%%%%%%%%%%%%%%%%%%%%%%%%%%%%%%%%%%%%%%%%%%%
%\begin{frame}
%\frametitle{Introducci\'on a la Cosmolog\'ia}
%La cosmolog\'ia es la ciencia que estudia el Universo como un todo. El principio cosmol\'ogico es la base de cualquier modelo de evoluci\'on. Este principio aclara que \textbf{el Universo es homog\'eneo e is\'otropo en un espacio tridimensional.}
%\end{frame}
%%%%%%%%%%%%%%%%%%%%%%%%%%%%%%%%%%%%%%%%%%%%%%%%%%%%%%%%%%%%%%%%%%%%%%%%%%%%%%%%%%%%%%%%%%%%%%%%%%%%%%%%%%%%%%%%%%%%%%%%

%Basado en la Teor\'ia del Big Bang y en la Relatividad, la intuici\'on indicar\'ia que si, el Universo se expande, pero que en alg\'un punto en el tiempo, esta expansi\'on deber\'ia reducirse y detenerse, toda la materia tendr\'ia que recolapsar sobre si misma, debido a la autogravitaci\'on.


\begin{frame}
\frametitle{El Universo se expande}
En 1927, Edwin Hubble en un estudio concluye que las galaxias en el Universo est\'an en expansi\'on, a raz\'on
\begin{equation}
\vec{v} = H \vec{r},\label{eqn 1}
\end{equation}
donde $H(t)$ es el \textit{par\'ametro de Hubble}, en coordenadas com\'oviles
\begin{equation}
\vec{r}=a(t)\vec{x},\label{eqn 2}
\end{equation}
donde $a(t)$ es el \textit{factor de escala} del Universo. Entonces
\begin{equation}
\vec{v}=\vec{\dot{r}} = H\vec{r},\label{eqn 3}
\end{equation}
\begin{equation}
\frac{d}{d t}(a\vec{x}) = Ha\vec{x},\label{eqn 4}
\end{equation}
entonces
\begin{equation}
H(t) = \frac{\dot{a}(t)}{a(t)}\label{eqn 5} 
\end{equation}


\end{frame}
%%%%%%%%%%%%%%%%%%%%%%%%%%%%%%%%%%%%%%%%%%%%%%%%%%%%%%%%%%%%%%%%%%%%%%%%%%%%%%%%%%%%%%%%%%%%%%%%%%%%%%%%%%%%%%%%%%%%%%%%

%\begin{frame}
%La luz de las galaxias tiene diferentes caracter\'isticas espectrales propias de los \'atomos que la componen, al examinar l\'ineas de emisi\'on se concluye que se desplazan al rojo del especto. Este cambio indica que todas las galaxias se alejan y se define como 
%\begin{equation}
%z \equiv \frac{\lambda-\lambda_{0}}{\lambda_{0}},\label{eqn 6}
%\end{equation}
%la velocidad relativa entre galaxias viene dada por 
%\begin{equation}
%d\vec{v} = H d\vec{r} = \frac{\dot{a}}{a}cdt = c \frac{da}{a}\label{eqn 7}
%\end{equation}
%donde $c$ es la velocidad de la luz de un fot\'on emitido por una galaxia. Si se escribe  el cambio en la longitud de onda emitida como $d\lambda = \lambda - \lambda_{0}$\label{eqn 8}, sustituyendo en la ecuaci\'on (\ref{eqn 7}) se obtiene 
%\begin{equation}
%c\frac{da}{a} = d \vec{v} = cz = c\frac{d\lambda}{\lambda}.%\label{eqn 8}
%\end{equation}
%\end{frame}
%%%%%%%%%%%%%%%%%%%%%%%%%%%%%%%%%%%%%%%%%%%%%%%%%%%%%%%%%%%%%%%%%%%%%%%%%%%%%%%%%%%%%%%%%%%%%%%%%%%%%%%%%%%%%%%%%%%%%%%%
\begin{frame}
\frametitle{Ecuaciones de Einstein}
Indican la curvatura del espacio-tiempo del Universo debido a la presencia de materia-energ\'ia en el mismo. 

%Introduciendo un espacio-tiempo mediante la ecuaci\'on de la m\'etrica
%\begin{equation}
%ds^{2} = g_{\alpha\beta}dx^{\alpha}dx^{\beta},\label{eqn 9}
%\end{equation}
%se obtiene
%\begin{equation}
%R_{\alpha\beta} - \frac{1}{2} R g_{\alpha\beta} + \Lambda g_{\alpha\beta} = 0,\label{eqn 10}
%\end{equation}
%$R{\alpha\beta}$ son el tensor y escalar de Ricci respectivamente, $g_{\alpha\beta}$ es el tensor m\'etrico y $\Lambda$ es la constante cosmol\'ogica. En presencia de materia y energ\'ia, la ecuaci\'on (\ref{eqn 10}) es
\begin{equation}
R_{\alpha\beta} - \frac{1}{2} R g_{\alpha\beta} + \Lambda g_{\alpha\beta} = \kappa T_{\alpha\beta}, \label{eqn 11}
\end{equation}
$R{\alpha\beta}$ son el tensor y escalar de Ricci respectivamente, $g_{\alpha\beta}$ es el tensor m\'etrico, $\Lambda$ es la constante cosmol\'ogica y $T_{\alpha\beta}$ es el tensor de energ\'ia-momento y $\kappa = 8\pi G/c^{4}$.
\begin{figure}
 \centering
\includegraphics[width=0.35\textwidth, height=0.25\textwidth]{./Figuras/RG}
\end{figure}


\end{frame}
%%%%%%%%%%%%%%%%%%%%%%%%%%%%%%%%%%%%%%%%%%%%%%%%%%%%%%%%%%%%%%%%%%%%%%%%%%%%%%%%%%%%%%%%%%%%%%%%%%%%%%%%%%%%%%%%%%%%%%%%
\begin{frame}
\frametitle{M\'etrica de Friedmann-Lema\^itre-Robertson-Walker}
Dados dos eventos en el espacio--tiempo, $(t,r,\theta , \phi)$ y en $(t +dt, r + dr, \theta + d\theta, \phi + d\phi)$. La separaci\'on espacio-temporal entre estos dos eventos es
\begin{equation}
ds^{2}=-c^{2}dt^{2} + dr^{2} +r^{2}d\Omega^{2},\label{eqn 12}
\end{equation}
donde
\begin{equation*}
d\Omega^{2} \equiv d\theta^{2} + \sin^{2}\theta d\phi^{2}.
\end{equation*}
Esta ecuaci\'on se conoce como \textit{m\'etrica de Minkowski}, la cual describe espacios planos. Al a\~nadir gravedad, la m\'etrica cambia a la siguiente
\begin{equation}
ds^{2}
=
-c^{2}dt^{2} + a^{2}(t)
\left[
\frac{dr^{2}}{1-k r^{2}} + r^{2}d\Omega^{2}
\right],\label{eqn 13}
\end{equation}
donde $k > 0$, $k = 0$ o $k < 0$ describen espacios abiertos, planos o cerrados, respectivamente.
\end{frame}
%%%%%%%%%%%%%%%%%%%%%%%%%%%%%%%%%%%%%%%%%%%%%%%%%%%%%%%%%%%%%%%%%%%%%%%%%%%%%%%%%%%%%%%%%%%%%%%%%%%%%%%%%%%%%%%%%%%%%%%%
\begin{frame}
La ecuaci\'on (\ref{eqn 13}) se conoce como \textit{m\'etrica de Robertson--Walker} \begin{figure}
\centering
    \includegraphics[width=0.4\textwidth]{./Figuras/geometry}
    \includegraphics[width=0.4\textwidth]{./Figuras/ScaleFactor}
\end{figure}
El Universo puede tener diferentes geometr\'ias, dependiendo su curvatura $k$, al d\'ia de hoy se observa que es casi plano.

Se necesita un tensor de energ\'ia-momento homog\'eneo e isotr\'opico para resolver las ecuaciones de Einstein (\ref{eqn 11}). Este es el de un fluido perfecto, que para la m\'etrica de FLRW, es diagonal
%\begin{equation}
%T_{\mu \nu} =
%(\rho + p)u_{\mu}u_{\nu} + pg_{\mu \nu},\label{eqn 14}
%\end{equation}
%donde $\rho$ es la densidad de masa propia del fluido, $u_{\mu}$ es la cuadrivelocidad y $p$ es la presi\'on $(p > 0)$ o la tensi\'on $(p < 0)$. Para la m\'etrica de la ecuaci\'on (\ref{eqn 13}) este tensor es diagonal
\begin{equation}
T_{\mu \nu} = \textup{diag}(-\rho, p, p, p).\label{eqn 15}
\end{equation}
\end{frame}
%%%%%%%%%%%%%%%%%%%%%%%%%%%%%%%%%%%%%%%%%%%%%%%%%%%%%%%%%%%%%%%%%%%%%%%%%%%%%%%%%%%%%%%%%%%%%%%%%%%%%%%%%%%%%%%%%%%%%%%%

%%%%%%%%%%%%%%%%%%%%%%%%%%%%%%%%%%%%%%%%%%%%%%%%%%%%%%%%%%%%%%%%%%%%%%%%%%%%%%%%%%%%%%%%%%%%%%%%%%%%%%%%%%%%%%%%%%%%%%%%
\begin{frame}
Las ecuaciones de Friedmann se encuentran a partir de las ecuaciones de campo de Einstein para la m\'etrica FLRW. Recordando la relaci\'on del par\'ametro de Hubble $H(t)=\dot{a}/a$ e insertando la ecuaci\'on (\ref{eqn 15}) en (\ref{eqn 11}) y tomando $\Lambda = 0$, $c=1$
\begin{equation*}
 3\frac{\dot{a}^{2} + k}{a^{2}}  = 8 \pi G \rho
 \end{equation*}
\begin{equation} 
-2\frac{\ddot{a}}{a} - \frac{\dot{a}^{2} + k}{a^{2}}= 8 \pi G p.\label{eqn 16}
\end{equation}
Al combinar ambas ecuaciones se obtiene
\begin{equation}
\frac{\ddot{a}}{a} = -\frac{4 \pi G}{3}(\rho + 3p), \label{eqn 17}
\end{equation}
Estas ecuaciones describen la cantidad de materia contenida en el Universo y determinan su geometr\'ia.
\end{frame}
%%%%%%%%%%%%%%%%%%%%%%%%%%%%%%%%%%%%%%%%%%%%%%%%%%%%%%%%%%%%%%%%%%%%%%%%%%%%%%%%%%%%%%%%%%%%%%%%%%%%%%%%%%%%%%%%%%%%%%%%
%\begin{frame}
%\frametitle{$\Lambda$CDM}
%Aunque la primera evidencia de materia oscura fue descubierta en la d\'ecada de 1930, no fue hasta la d\'ecada de 1980 que los astr\'onomos se convencieron de que \'esta es el componente responsable que mantiene unidas a las galaxias y los c\'umulos de galaxias de manera gravitacional.

%El modelo \textit{Lambda Cold Dark Matter} ($\Lambda$CDM) es una parametrizaci\'on del modelo del Big Bang. Tambi\'en conocido como el ``modelo est\'andar de la cosmolog\'ia'' se  fundamenta,  principalmente, sobre las siguientes bases te\'oricas y experimentales.
%\end{frame}
%%%%%%%%%%%%%%%%%%%%%%%%%%%%%%%%%%%%%%%%%%%%%%%%%%%%%%%%%%%%%%%%%%%%%%%%%%%%%%%%%%%%%%%%%%%%%%%%%%%%%%%%%%%%%%%%%%%%%%%%
%%%%%%%%%%%%%%%%%%%%%%%%%%%%%%%%%%%%%%%%%%%%%%%%%%%%%%%%%%%%%%%%%%%%%%%%%%%%%%%%%%%%%%%%%%%%%%%%%%%%%%%%%%%%%%%%%%%%%%%%
\begin{frame}
\frametitle{Materia Oscura}
\begin{itemize}
 \setlength\itemsep{1em}
\item 1933 Fritz Zwicky aplica el teorema del Virial al C\'umulo de Coma
\item La materia visible no puede explicar las velocidades de rotaci\'on del c\'umulo
\item Materia Faltante u ``oscura''
\end{itemize}
\centering
\includegraphics[width=0.5\textwidth]{./Figuras/Zwicky}
\end{frame}
%%%%%%%%%%%%%%%%%%%%%%%%%%%%%%%%%%%%%%%%%%%%%%%%%%%%%%%%%%%%%%%%%%%%%%%%%%%%%%%%%%%%%%%%%%%%%%%%%%%%%%%%%%%%%%%%%%%%%
\begin{frame}
\begin{itemize}
 \setlength\itemsep{2em}
\item 1980 Vera Rubin hace observaciones en galaxias espirales
\item Las curvas de rotaci\'on de las estrellas no tienen comportamiento Kepleriano 
\item Evidencia Observacional de materia oscura
\end{itemize}
\centering
\includegraphics[width=0.4\textwidth]{./Figuras/M33Rotation}
\includegraphics[width=0.4\textwidth]{./Figuras/Rubin}
\end{frame}
%%%%%%%%%%%%%%%%%%%%%%%%%%%%%%%%%%%%%%%%%%%%%%%%%%%%%%%%%%%%%%%%%%%%%%%%%%%%%%%%%%%%%%%%%%%%%%%%%%%%%%%%%%%%%%%%%%%%%%%%%%%%%%%
\begin{frame}
\textbf{Contenido estimado del Universo}
\begin{figure}
\centering
\includegraphics[width=0.6\textwidth]{./Figuras/materia-energia}
\end{figure}
\end{frame}
%%%%%%%%%%%%%%%%%%%%%%%%%%%%%%%%%%%%%%%%%%%%%%%%%%%%%%%%%%%%%%%%%%%%%%%%%%%%%%%%%%%%%%%%%%%%%%%%%%%%%%%%%%%%%%%%%%%%%%%%%%%%%%%%%%%%%%%%%%%%%%%%%%%%%%%%%%%%%%
\begin{frame}
\frametitle{$\Lambda$CDM}
\begin{itemize}
 \setlength\itemsep{2em}
\item Un marco te\'orico basado en la teor\'ia general de la relatividad, que proporciona la teor\'ia del campo gravitatorio en escalas cosmol\'ogicas.
\item El principio cosmol\'ogico.
\item El modelo de fluidos, que considera a las galaxias como constituyentes b\'asicos del Universo.
\item La Ley de Hubble.
\item La Radiaci\'on del Fondo C\'osmico de Microondas.
%\item La determinaci\'on de la abundancia relativa de elementos primigenios $^{1}$H, $^{2}$D, $^{3}$He, $^{4}$He y $^{7}$Li formados en las reacciones nucleares en la \'epoca de Big Bang Nucleos\'intesis.
%\item El an\'alisis de la estructura a gran escala del Universo, mediante experimentos como el SDSS, que atestiguan la homogeneidad y ayudan a la determinaci\'on de los distintos par\'ametros del modelo est\'andar.
\end{itemize}
\end{frame}
%%%%%%%%%%%%%%%%%%%%%%%%%%%%%%%%%%%%%%%%%%%%%%%%%%%%%%%%%%%%%%%%%%%%%%%%%%%%%%%%%%%%%%%%%%%%%%%%%%%%%%%%%%%%%%%%%%%%%%%%
\begin{frame}
\begin{itemize}
 \setlength\itemsep{2em}
\item Perturbaciones a la densidad. Tambi\'en conocidas como fluctuaciones de densidad o fluctuaciones cu\'anticas.

\item La \textit{Inflaci\'on}, una expansi\'on acelerada y que explica la planitud y la homogeneidad actuales del Universo.

%\item El Hot Big Bang, origen extremadamente caliente que da lugar a BBN.

\item La \textit{constante cosmol\'ogica} $\Lambda$, que Einstein introdujo en las ecuaciones de la relatividad general
\begin{equation}
H^{2}(t) = \left(\frac{\dot{a}}{a}\right)^{2}
= 
\frac{8 \pi G}{3} - \frac{k}{a^{2}} + \frac{\Lambda}{3}.\label{eqn 18}
\end{equation}
\item La \textit{materia oscura fr\'ia}, Cold Dark Matter (CDM). Un tipo de materia que act\'ua de forma gravitatora, \textbf{oscura} $\rightarrow$ sin interacci\'on con otro tipo de materia, \textbf{fr\'ia} $\rightarrow$ Velocidad no relativista.
\end{itemize}

\end{frame}
%%%%%%%%%%%%%%%%%%%%%%%%%%%%%%%%%%%%%%%%%%%%%%%%%%%%%%%%%%%%%%%%%%%%%%%%%%%%%%%%%%%%%%%%%%%%%%%%%%%%%%%%%%%%%%%%%%%%%%%%
\begin{frame}
\textbf{CMB y L\'inea temporal del Universo}
\begin{figure}
\centering
\includegraphics[width=0.4\textwidth]{./Figuras/planck}
\includegraphics[width=0.4\textwidth]{./Figuras/BBCosmology}
\end{figure}
La Radiaci\'on del Fondo C\'osmico de Microondas (CMB) es un tipo de radiaci\'on que data de $\sim$ 380,000 a\~nos despu\'es del Big Bang. Se estima que la temperatura de esta radiaci\'on ronda los $2.785 \pm 0.005$ K.
\end{frame}
%%%%%%%%%%%%%%%%%%%%%%%%%%%%%%%%%%%%%%%%%%%%%%%%%%%%%%%%%%%%%%%%%%%%%%%%%%%%%%%%%%%%%%%%%%%%%%%%%%%%%%%%%%%%%%%%%%%%%%%%
\begin{frame}
\frametitle{Problemas con $\Lambda$CDM}
\begin{itemize}
 \setlength\itemsep{1em}
\item Problema CUSP-CORE $\rightarrow$ Perfiles de densidad de materia oscura no concuerdan con observaciones
\item Sat\'elites faltantes $\rightarrow$ CDM predice $100-1000$ galaxias sat\'elites para una galaxia del tama\~no de la V\'ia L\'actea. Solo se conocen 23 
\item Entre otros...
\end{itemize}
\begin{figure}
\includegraphics[width=0.6\textwidth, height=0.3\textwidth]{./Figuras/CORECUSP}
\includegraphics[width=0.3\textwidth, height=0.3\textwidth]{./Figuras/Satellites}
\end{figure}
\end{frame}
%%%%%%%%%%%%%%%%%%%%%%%%%%%%%%%%%%%%%%%%%%%%%%%%%%%%%%%%%%%%%%%%%%%%%%%%%%%%%%%%%%%%%%%%%%%%%%%%%%%%%%%%%%%%%%%%%%%%%%%%SFDM
\begin{frame}
\frametitle{Scalar Field Dark Matter (SFDM)}
\begin{itemize}
 \setlength\itemsep{2em}
\item La materia oscura se considera como campo escalar $\Phi$ con potencial escalar $V(\Phi)$.
\item M\'inimamente acoplado a la gravedad, a ciertas temperaturas se comporta como Condensado de Bose-Einstein (BEC).
\item $M_{\chi} \sim 10^{-22}$eV 
\item $\lambda \sim \mathcal{O}(kpc) \sim$ tama\~no de una galaxia.
\item Posible soluci\'on a problemas de $\Lambda$CDM.
\end{itemize}
\end{frame}
%%%%%%%%%%%%%%%%%%%%%%%%%%%%%%%%%%%%%%%%%%%%%%%%%%%%%%%%%%%%%%%%%%%%%%%%%%%%%%%%%%%%%%%%%%%%%%%%%%%%%%%%%%%%%%%%%%%%%%%%
%\begin{frame}
%\frametitle{Campos escalares}
%%%PONER CAMPO ESCALAR TEORIA
%En el modelo de Campo Escalar, se propone que los halos gal\'acticos se forman de condensados de Bose-Einstein de un campo escalar (SF) cuyo bos\'on tiene una masa ultra ligera del orden de $m \sim 10^{-22}$eV. De este valor se sigue que la temperatura cr\'itica de condensaci\'on $T_{c} \sim 1/m^{5/3} \sim $ TeV, es muy alta, por lo tanto, se forman semillas de Condensados de Bose-Einstein (CBE) en \'epocas tempranas en el Universo.
%\end{frame}
%%%%%%%%%%%%%%%%%%%%%%%%%%%%%%%%%%%%%%%%%%%%%%%%%%%%%%%%%%%%%%%%%%%%%%%%%%%%%%%%%%%%%%%%%%%%%%%%%%%%%%%%%%%%%%%%%%%%%%%%
%\begin{frame}
%Recordando las ecuaciones de FLRW, el tensor energ\'ia-momento \textbf{T} para un campo escalar, la densidad de energ\'ia escalar $T_{0}^{0}$ y la presi\'on escalar $T_{j}^{i}$ estar\'an dadas por
%\begin{equation}
%T_{0}^{0}=-\rho_{\Phi_{0}}=-\left(\frac{1}{2}\dot{\Phi}_{0}^{2} + V \right),\label{eqn 19}
%\end{equation}
%y
%\begin{equation}
%T_{j}^{i}=P_{\Phi_{0}}=\left(\frac{1}{2} \dot{\Phi}_{0}^{2}-V \right)\delta_{j}^{i},\label{eqn 20}
%\end{equation}
%donde el punto se entiende como la derivada respecto al tiempo cosmol\'ogico y $\delta_{j}^{i}$ es la delta de Kronecker.
%\end{frame}
%%%%%%%%%%%%%%%%%%%%%%%%%%%%%%%%%%%%%%%%%%%%%%%%%%%%%%%%%%%%%%%%%%%%%%%%%%%%%%%%%%%%%%%%%%%%%%%%%%%%%%%%%%%%%%%%%%%%%%%%
%\begin{frame}
%la Ecuaci\'on de Estado para el campo escalar es $p_{\Phi_{0}}=\omega_{\Phi_{0}}\rho_{\Phi_{0}}$ con
%\begin{equation}
%\omega_{\Phi_{0}} = \frac{\frac{1}{2}\dot{\Phi}_{0}^{2}-V}{\frac{1}{2}\dot{\Phi}_{0}^{2}+V}.\label{eqn 21}
%\end{equation}
%Se definen nuevas variables adimensionales
%\begin{equation}
%x\equiv \frac{\kappa}{\sqrt{6}}\frac{\Phi_{0}}{H}, \;\;\; u\equiv\frac{\kappa}{\sqrt{3}}\frac{\sqrt{V}}{H}\label{eqn 22}
%\end{equation}
%donde $\kappa^{2}\equiv 8\pi G$ y $H \equiv \dot{a}/a$ es el par\'ametro de Hubble. Se toma el potencial escalar como $V = m^{2}\Phi^{2}/2\hbar^{2} + \lambda\Phi^{4}/4$, si se toma $c = 1$, para un bos\'on ultra ligero se tendr\'a que $\mu_{\Phi} \sim 10^{-22} $eV.
%\end{frame}
%%%%%%%%%%%%%%%%%%%%%%%%%%%%%%%%%%%%%%%%%%%%%%%%%%%%%%%%%%%%%%%%%%%%%%%%%%%%%%%%%%%%%%%%%%%%%%%%%%%%%%%%%%%%%%%%%%%%%%%%
%\begin{frame}
%\frametitle{Aproximaci\'on hidrodin\'amica}
%En esta aproximaci\'on, se hace una transformaci\'on para %resolver las ecuaciones de Friedmann de manera anal\'itica con %la condici\'on $H<<m$. 
%Se toma el potencial escalar como $V = m^{2}\Phi^{2}/2\hbar^{2} + \lambda\Phi^{4}/4$. As\'i, para el bos\'on ultra ligero se tiene que $m \sim 10^{-22}$ eV. $\Phi_{0}$ se expresa en t\'erminos de nuevas variables $S$ y $\rho_{0}$, donde $S$ es constante en el fondo y $\rho_{0}$ ser\'a la densidad de energ\'ia del fluido tambi\'en en esta regi\'on, así el campo en dicha región se expresa como
%\begin{equation}
%\Phi_{0} = (\psi_{0}e^{-imt/\hbar} + \psi_{0}^{*}e^{imt/\hbar}),\label{eqn 23}
%\end{equation}
%donde
%\begin{equation}
%\psi_{0}(t) = \sqrt{\rho_{0}(t)}e^{iS/\hbar},\label{eqn 24}
%\end{equation}
%As\'i, el campo escalar en la regi\'on del fondo del Universo se puede expresar como 
%\begin{equation}
%\Phi_{0}=2\sqrt{\rho_{0}}\cos(S-mt/\hbar),\label{eqn 25}
%\end{equation}
%con esto se obtiene
%\begin{equation}
%\dot{\Phi}_{0}^{2} = \rho_{0} 
%\left[
%\frac{\dot{\rho}_{0}}{\rho_{0}}\cos(S-mt/\hbar) 
%- 2(\dot{S}-mt/\hbar)\sin(S-mt/\hbar)
%\right]^{2}.\label{eqn 26}
%\end{equation}
%Observe que el principio de incertidumbre implica que $m \Delta t \sim \hbar$, y que para el fondo en el caso no relativista se cumple la relación $\dot{S}/m \sim 0$.
%\end{frame}
%%%%%%%%%%%%%%%%%%%%%%%%%%%%%%%%%%%%%%%%%%%%%%%%%%%%%%%%%%%%%%%%%%%%%%%%%%%%%%%%%%%%%%%%%%%%%%%%%%%%%%%%%%%%%%%%%%%%%%%%
\begin{frame}
\frametitle{Aproximaci\'on hidrodin\'amica y perturbaciones}
%El campo escalar tiene oscilaciones intensas desde el inicio, estas oscilaciones se transmiten a las fluctuaciones que crecen de manera r\'apida. 
%Una perturbaci\'on en cualquier cantidad es la diferencia entre su valor correspondiente en un evento en el espacio--tiempo real y su correspondiente valor de ``fondo'', as\'i, 
Para el campo escalar, se propone la siguiente perturbaci\'on
\begin{equation}
\Phi = \Phi_{0}(t) + \delta\Phi(\vec{x},t),\label{eqn 27}
\end{equation}
que al insertar en la ecuaci\'on de Klein-Gordon, se tiene ($\dot{\phi}=0$)
\begin{equation}
\delta\ddot{\Phi} + 3H\delta\dot{\Phi}
- \frac{1}{a^{2}}\vec{\nabla}^{2}\delta\Phi
+V_{,\Phi\Phi}\delta\Phi +2V_{,\Phi}\phi = 0.\label{eqn 28}
\end{equation}
El campo escalar perturbado $\delta\Phi$ se expresa en t\'erminos de $\Psi$ como sigue
\begin{equation}
\delta\Phi = \Psi e^{-imt/\hbar} +\Psi^{*}e^{imt/\hbar},\label{eqn 29}
\end{equation}
\end{frame}
%%%%%%%%%%%%%%%%%%%%%%%%%%%%%%%%%%%%%%%%%%%%%%%%%%%%%%%%%%%%%%%%%%%%%%%%%%%%%%%%%%%%%%%%%%%%%%%%%%%%%%%%%%%%%%%%%%%%%%%%%%%%%%%%%%%%%%%%%%%%%%%%%%%%%%%%%%%%%%%%%%%%%%%%%%%%%%%%%%%%%%%%%%%%%%%%%%%%%%%%%%%%%%%%%%%%%%%%%%%%%%%%%%%%%%%%%%%%%%%%
\begin{frame}
Con esta ecuaci\'on y la expresi\'on del potencial del campo escalar, la ecuaci\'on (\ref{eqn 28}) se convierte en 
\begin{equation}
-i\hbar(\dot{\Psi}+\frac{3}{2}H\Psi) + \frac{\hbar^{2}}{2m}(\Box \Psi +9\lambda|\Psi|^{2}\Psi) + m\phi\Psi = 0,\label{eqn 30}
\end{equation}
donde $\Box$ se define como 
\begin{equation}
\Box = \frac{d^{2}}{d t^{2}} + 3H\frac{d}{d t} - \frac{1}{a^{2}}\vec{\nabla}^{2}\label{eqn 31}
\end{equation}
%Para entender la naturaleza hidrodin\'amica de este modelo, 
Se hace una aproximaci\'on utilizando una transformaci\'on de Madelung, la cual conecta la teor\'ia de campos y las funciones de onda de los condensados
\end{frame}
%%%%%%%%%%%%%%%%%%%%%%%%%%%%%%%%%%%%%%%%%%%%%%%%%%%%%%%%%%%%%%%%%%%%%%%%%%%%%%%%%%%%%%%%%%%%%%%%%%%%%%%%%%%%%%%%%%%%%%%%
\begin{frame}
\begin{equation}
\Psi=\sqrt{\hat{\rho}} e^{iS},\label{eqn 32}
\end{equation}
con amplitud $\hat{\rho}=\rho/m=\hat{\rho}(\vec{x},t)$ y fase $S=S(\vec{x},t)$ reales. Se satisface la condici\'on $|\Psi|^{2}=\Psi\Psi^{*}= \hat{\rho}$. Sustituyendo en la ecuaci\'on (\ref{eqn 30}), se obtiene
\begin{equation}
\dot{\hat{\rho}} + 3H\hat{\rho}
-\frac{\hbar}{m}\hat{\rho}\Box S 
+\frac{\hbar}{a^{2}m}\vec{\nabla}S\vec{\nabla}\hat{\rho}
-\frac{\hbar}{m}\hat{\rho}\dot{S}=0,\label{eqn 33}
\end{equation}
y
\begin{equation}
\hbar \dot{S}/m + \omega\hat{\rho}
+ \phi
+ \frac{\hbar^{2}}{2m^{2}}\left(\frac{\Box\sqrt{\hat{\rho}}}{\sqrt{\hat{\rho}}}\right)
+ \frac{\hbar^{2}}{2a^{2}}[\vec{\nabla}(S/m)]^{2}
- \frac{\hbar^{2}}{2}(\dot{S}/m)^{2} = 0.\label{eqn 34}
\end{equation}
\end{frame}%%%%%%%%%%%%%%%%%%%%%%%%%%%%%%%%%%%%%%%%%%%%%%%%%%%%%%%%%%%%%%%%%%%%%%%%%%%%%%%%%%%%%%%%%%%%%%%%%%%%%%%%%%%
\begin{frame}
Tomando el gradiente de las ecuaciones (\ref{eqn 33}), (\ref{eqn 34}) dividiendo por $a$ y utilizando la definici\'on 
\begin{equation}
\vec{v}\equiv \frac{\hbar}{ma}\vec{\nabla}S\label{eqn 35}
\end{equation}
se obtiene
\begin{equation}
\dot{\hat{\rho}} + 3H\hat{\rho} - \frac{\hbar}{m}\hat{\rho}\Box S 
+ \frac{1}{a}\vec{v}\vec{\nabla}\hat{\rho} - \frac{\hbar}{m}\hat{\rho}\dot{S} = 0,\label{eqn 36}
\end{equation}
\begin{equation*}
\dot{\vec{v}} + H\vec{v} + \frac{1}{2a\hat{\rho}}\vec{\nabla}p 
+ \frac{1}{a}\vec{\nabla}\phi + \frac{\hbar^{2}}{2m^{2}a}\vec{\nabla}
\left(\frac{\Box\sqrt{\hat{\rho}}}{\sqrt{\hat{\rho}}}\right) 
\end{equation*}  
\begin{equation}
+\frac{1}{a}(\vec{v}\cdot\vec{\nabla})\vec{v}-\hbar(\dot{\vec{v}}+H\vec{v})(\dot{S}/m) = 0\label{eqn 37}
\end{equation}
donde se ha definido $\omega = 9\hbar^{2}\lambda/2m^{2}$ y $p= \omega\hat{ \rho}^{2}$. 
\end{frame}
%%%%%%%%%%%%%%%%%%%%%%%%%%%%%%%%%%%%%%%%%%%%%%%%%%%%%%%%%%%%%%%%%%%%%%%%%%%%%%%%%%%%%%%%%%%%%%%%%%%%%%%%%%%
\begin{frame}
Al ignorar t\'erminos cuadr\'aticos y derivadas temporales de segundo orden en (\ref{eqn 36}) y (\ref{eqn 37}) se obtiene
\begin{equation}
\frac{\partial \hat{\rho}}{\partial t} +
\vec{\nabla}\cdot(\hat{\rho} \vec{v}) + 3H\hat{\rho}=0,\label{eqn 38}
\end{equation}
\begin{equation}
\frac{\partial \vec{v}}{\partial t} + H \vec{v}
-\frac{\hbar^{2}}{2m^{2}}\vec{\nabla}\left(\frac{1}{2\hat{\rho}}\vec{\nabla}^{2}\hat{\rho}\right) + \omega\vec{\nabla}\hat{\rho} +\vec{\nabla}^{2}\phi,\label{eqn 39}
\end{equation}
\begin{equation}
\vec{\nabla}^{2}\phi = 4\pi G\hat{\rho},\label{eqn 40}
\end{equation}
Las ecuaciones (\ref{eqn 36}) y (\ref{eqn 37}) conducen al an\'alogo de las ecuaciones de Euler para fluidos cl\'asicos, con la diferencia de la existencia de un t\'ermino cu\'antico $Q$, definido por 
\begin{equation}
Q = \frac{\hbar^{2}}{2m^{2}}\frac{\Box\sqrt{\hat{\rho}}}{\sqrt{\hat{\rho}}},\label{eqn 1.78}
\end{equation}
este \'ultimo t\'ermino, puede describir una fuerza o ``presi\'on '' negativa de naturaleza cu\'antica.
\end{frame}
%%%%%%%%%%%%%%%%%%%%%%%%%%%%%%%%%%%%%%%%%%%%%%%%%%%%%%%%%%%%%%%%%%%%%%%%%%%%%%%%%%%%%%%%%%%%%%%%%%%%%%%%%%%%%%%%%%%%%%%%
\begin{frame}
\frametitle{Simulaciones de $N$-Cuerpos}
%La evoluci\'on de estructura se aproxima con aglomeramiento gravitacional  no lineal a partir de condiciones iniciales espec\'ificas de part\'iculas de materia oscura y puede refinarse introduciendo efectos de din\'amica de gases, procesos qu\'imicos, transferencia radiativa y otros procesos astrof\'isicos.
La evoluci\'on no lineal de la materia oscura utiliza la ecuaci\'on de Boltzmann no colisional en coordenadas com\'oviles acoplada con la ecuaci\'on de Poisson
\begin{equation}
\frac{\partial f}{\partial t} + \vec{v}\cdot\frac{\partial f}{\partial r} + \frac{\vec{F}}{m}\cdot\frac{\partial f}{\partial v}=0,\label{eqn 39}
\end{equation}
$f= f(\vec{r}, \vec{p}, t)$ es funci\'on de la distribuci\'on de la densidad del fluido. %es una funci\'on de distribuci\'on de la densidad, $\vec{v}$ es la velocidad, $\vec{r}$ es la posici\'on, $\vec{F}$ es la fuerza y $m$ la masa que describen completamente al fluido.
En el caso de que esta fuerza $\vec{F}$ se derive de un potencial, tal que 
\begin{equation}
\vec{F} = -m\nabla\Phi,\label{eqn 40}
\end{equation}
donde $m$ es la masa de la part\'icula del sistema. Sustituyendo (\ref{eqn 40}) en (\ref{eqn 39}) se encuentra
\begin{equation}
\frac{\partial f}{\partial t} + \vec{v}\cdot\frac{\partial f}{\partial r} - \vec\nabla\Phi\cdot\frac{\partial f}{\partial v}=0\label{eqn2.3}
\end{equation} 
%Este potencial $\Phi$ debe satisfacer la ecuación de Poisson
%\begin{equation}
%\nabla^{2} \Phi (\vec{r},t) = 4\pi \int_{S} f(\vec{r},\vec{v},t)d^{3}v\label{eqn2.4}
%\end{equation}
%donde $S$ es todo el espacio y $f$ se define mediante la siguiente expresión
%\begin{equation}
%f = f(\vec{r},\vec{v},t)d^{3}v d^{3}r\label{eqn 2.5}
%\end{equation}

\end{frame}
%%%%%%%%%%%%%%%%%%%%%%%%%%%%%%%%%%%%%%%%%%%%%%%%%%%%%%%%%%%%%%%%%%%%%%%%%%%%%%%%%%%%%%%%%%%%%%%%%%%%%%%%%%%%%%%%%%%%%%%%
%\begin{frame}
%\frametitle{Fluidos sin Colisi\'on Autogravitantes}
%\end{frame}
%%%%%%%%%%%%%%%%%%%%%%%%%%%%%%%%%%%%%%%%%%%%%%%%%%%%%%%%%%%%%%%%%%%%%%%%%%%%%%%%%%%%%%%%%%%%%%%%%%%%%%%%%%%%%%%%%%%%%%%%
\begin{frame}
Dadas las coordenadas iniciales $\vec{r}_{init}$ y velocidades $\vec{v}_{init}$ de $N$ part\'iculas con masa en el momento $t = t_{init}$, encontrar sus coordenadas $\vec{r}$ y velocidades $\vec{v}$ en el siguiente instante $t = t_{next}$. Si $\vec{r}_{i}$ y $m_{i}$ es la coordenada y masa para cada part\'icula, entonces las ecuaciones de movimiento son
\begin{equation}
\frac{d^{2}\vec{r}_{i}}{d t^{2}}=
-G \sum_{j=1, i \not= j}^{N} \frac{m_{j}(\vec{r}_{i}-\vec{r}_{j})}{|\vec{r}_{i}-\vec{r}_{j}|^{3}}, \label{eqn 42}
\end{equation}
\centering
\animategraphics[autoplay,loop,height=5cm]{9}{./Figuras/nbody}{0}{8}
%donde $G$ es la consante de gravitaci\'on universal.
\end{frame}
%%%%%%%%%%%%%%%%%%%%%%%%%%%%%%%%%%%%%%%%%%%%%%%%%%%%%%%%%%%%%%%%%%%%%%%%%%%%%%%%%%%%%%%%%%%%%%%%%%%%%%%%%%%%%%%%%%%%%%%%
\begin{frame}
Se introduce un suavizante de fuerza: la fuerza se hace m\'as d\'ebil (se suaviza) en distancias peque\~nas para evitar aceleraciones grandes
\begin{equation}
\frac{d^{2}\vec{r}_{i}}{d t^{2}}=
-G \sum_{j=1, i \not= j}^{N} \frac{m_{j}(\vec{r}_{i}-\vec{r}_{j})}{(\Delta\vec{r}_{ij}^{2} + \epsilon^{2})^{3/2}},\label{eqn 43}
\end{equation} 
donde $\Delta\vec{r}_{ij}^{2} = |\vec{r}_{i} - \vec{r}_{j}|$ y $\epsilon$ es el par\'ametro de suavizaci\'on o ``softening length''. Al considerar el modelo FLRW, la din\'amica de part\'iculas se describe con el Hamiltoniano
\begin{equation}
  H 
  = \sum_{i=1}^{N} \frac{\vec{p_{i}}^{2}}{2 m_{i} a^{2}(t)} 
  +
  \frac{1}{2} \sum_{i\not=1,i\not=j}^{N}\sum_{j=1,j\not=i}^{N}
  \frac{m_{i}m_{j}\Phi(\vec{x_{i}}-\vec{x_{j}})}{a(t)},\label{eqn 44}
\end{equation}
%donde $\vec{p_{k}}$ y $\vec{x_{k}}$ son los vectores de momento y posici\'on en el sistema de coordenadas com\'oviles y $a$ es el factor de escala.
\end{frame}
%%%%%%%%%%%%%%%%%%%%%%%%%%%%%%%%%%%%%%%%%%%%%%%%%%%%%%%%%%%%%%%%%%%%%%%%%%%%%%%%%%%%%%%%%%%%%%%%%%%%%%%%%%%%%%%%%%%%%%%%
\begin{frame}
\begin{itemize}
\item C\'alculo de fuerzas num\'erico  $\rightarrow$ \textit{TreePM algorithm}
\item Potencial de interacci\'on entre part\'iculas
\item Aplica transformada de Fourier para encontrar fuerza de interacci\'on
\item Proceso inverso
\end{itemize}
\begin{figure}
\centering
  \includegraphics[width=0.45\textwidth, height=0.4\textwidth]{./Figuras/BHAlgorithm}
  \includegraphics[width=0.4\textwidth]{./Figuras/PM}
  \label{fig 2.1}
\end{figure}
\end{frame}
%%%%%%%%%%%%%%%%%%%%%%%%%%%%%%%%%%%%%%%%%%%%%%%%%%%%%%%%%%%%%%%%%%%%%%%%%%%%%%%%%%%%%%%%%%%%%%%%%%%%%%%%%%%%%%%%%%%%%%%%
\begin{frame}
\frametitle{El C\'odigo}
\begin{itemize}
\item GADGET (GAlaxies with Dark matter and Gas intEracT), c\'odigo libre
\item Utiliza teor\'ia de $N$-cuerpos para simulaciones cosmol\'ogicas
\item Modificaci\'on para Campo escalar: \textbf{Axion-Gadget}
\item \url{https://wwwmpa.mpa-garching.mpg.de/~volker/gadget/}{\\ \textbf{GADGET-CODE}}
\item \url{https://github.com/liambx/Axion-Gadget}{\\ \textbf{Axion-Gadget}}
\end{itemize}
\begin{figure}
 \centering
  \includegraphics[width=0.6\textwidth]{./Figuras/Gadget}
\end{figure}
\end{frame}
%%%%%%%%%%%%%%%%%%%%%%%%%%%%%%%%%%%%%%%%%%%%%%%%%%%%%%%%%%%%%%%%%%%%%%%%%%%%%%%%%%%%%%%%%%%%%%%%%%%%%%%%%%%%%%%%%%%%%%%%
\begin{frame}
\frametitle{Axion-Gadget}
La naturaleza de FDM se describe mediante las ecuaciones de Schr\"odinger-Poisson
\begin{equation}
i\hbar \frac{d \Psi}{dt} 
=
-\frac{\hbar^{2}}{2m_{\chi}} \vec{\nabla}^{2}\Psi + m_{\chi}V\Psi,\label{eqn 3.1}
\end{equation}
y
\begin{equation}
\vec{\nabla}^{2}V = 4\pi G m_{\chi}|\Psi|^{2},\label{eqn 3.2}
\end{equation}
la funci\'on de onda se escribe como
\begin{equation}
\Psi = \sqrt{\frac{\rho}{m_{\chi}}}\exp(\frac{iS}{\hbar})\label{eqn 3.3}
\end{equation}
en t\'erminos de la densidad de n\'umero $\frac{\rho}{m_{\chi}}$, se define igualmente el momento lineal de la materia oscura como
\begin{equation}
\vec{\nabla}S = m_{\chi}\vec{v}.\label{eqn 3.4}
\end{equation}
\end{frame}

\begin{frame}
Al introducir la funci\'on de onda $\Psi$ en la ecuaci\'on (\ref{eqn 3.1}) y resolviendo las ecuaciones de Schr\"odinger-Poisson, se obtiene la ecuaci\'on de continuidad
\begin{equation}
\frac{d\rho}{dt} + \vec{\nabla}\cdot(\rho\vec{v}) = 0,\label{eqn 3.5}
\end{equation}
y la ecuaci\'on de conservaci\'on de momento
\begin{equation}
\frac{d\vec{v}}{dt} + (\vec{v}\cdot\vec{\nabla})\vec{v} 
=
-\vec{\nabla}(Q + V).\label{eqn 3.6}
\end{equation}
donde se define el potencial cu\'antico $Q$
\begin{equation}
Q 
=
-\frac{\hbar^{2}}{2m^{2}_{\chi}}\frac{\vec{\nabla}^{2}\sqrt{\rho}}{\sqrt{\rho}}.\label{eqn 3.7}
\end{equation}
En la simulaci\'on, la interacci\'on entre part\'iculas no es relativista, se utiliza el operador Laplaciano $\nabla$ en lugar del D'Alambertiano $\Box$.
\end{frame}
%%%%%%%%%%%%%%%%%%%%%%%%%%%%%%%%%%%%%%%%%%%%%%%%%%%%%%%%%%%%%%%%%%%%%%%%%%%%%%%%%%%%%%%%%%%%%%%%%%%%%%%%%%%%%%%%%%%%%%%%%
\begin{frame}
El efecto de la presi\'on cu\'antica, se describe con el Hamiltoniano sin el t\'ermino de gravedad
\begin{equation}
H = \int \frac{\hbar^{2}}{2 m_{\chi}} |\vec{\nabla}\Psi|^{2}d^{3}x
  = \int \frac{\rho}{2} |\vec{v}|^{2}d^{3}x + \int \frac{\hbar^{2}}{2 m_{\chi}} (\vec{\nabla}\sqrt{\rho})^{2}d^{3}x.\label{eqn 3.8}
\end{equation}
La energ\'ia cin\'etica en forma discreta, con \'indice $j$ para identificar a cada part\'icula se escribe como
\begin{equation}
T = \int \frac{\rho}{2} |\vec{v}|^{2}d^{3}x = \sum_{j} \frac{1}{2} m_{j} \left(\frac{d q_{j}}{dt}\right)^{2}, \label{eqn 3.9}
\end{equation}
la energ\'ia potencial
\begin{equation}
K_{p} = \int \frac{\hbar^{2}}{2 m_{\chi}} (\vec{\nabla}\sqrt{\rho})^{2}d^{3}x. \label{eqn 3.10}
\end{equation}
%Note que el término de la energía potencial $K_{p}$ no está discretizado aún. Esto se realizará más adelante.
Y el Lagrangiano
\begin{equation}
L = T - K_{p}
  =
\sum_{j} \frac{1}{2} m_{j} \left(\frac{d q_{j}}{dt}\right)^{2}
-
 \int \frac{\hbar^{2}}{2 m_{\chi}} (\vec{\nabla}\sqrt{\rho})^{2}d^{3}x,\label{eqn 3.11}  
\end{equation}
\end{frame}
%%%%%%%%%%%%%%%%%%%%%%%%%%%%%%%%%%%%%%%%%%%%%%%%%%%%%%%%%%%%%%%%%%%%%%%%%%%%%%%%%%%%%%%%%%%%%%%%%%%%%%%%%%%%%%%%%%%%%%%%
\begin{frame}
Las ecuaciones de Euler-Lagrange tendr\'an la forma
\begin{equation}
\frac{d}{dt}\frac{\partial L}{\partial \dot{q}_{j}} - \frac{\partial L }{\partial q_{j}} = 0 
\Rightarrow 
m_{j} \ddot{q}_{j} = - \frac{\partial K_{p}}{\partial q_{j}}.\label{eqn 3.12}
\end{equation}
\begin{wrapfigure}{r}{0.25\textwidth}
\centering
\includegraphics[width=0.30\textwidth, height=0.40\textwidth]{./Figuras/Density}
\end{wrapfigure}
Para una interacci\'on particle-mesh, la densidad de n\'umero para cada part\'icula individual se describe con funciones delta
\begin{equation}
\rho(\vec{r})
=
\sum_{i}m_{i}\delta(\vec{r}-\vec{r}_{i}).\label{eqn 3.13}
\end{equation}
donde
\begin{equation}
\delta (\vec{r}-\vec{r}_{i}) = 
\frac{2\sqrt{2}}{\lambda ^{3} \pi ^{3/2}} \exp \left(-\frac{2|\vec{r}-\vec{r}_{i}|^{2}}{\lambda ^{2}}\right).\label{eqn 3.14}
\end{equation}
\end{frame}
%%%%%%%%%%%%%%%%%%%%%%%%%%%%%%%%%%%%%%%%%%%%%%%%%%%%%%%%%%%%%%%%%%%%%%%%%%%%%%%%%%%%%%%%%%%%%%%%%%%%%%%%%%%%%%%%%%%%%%%%
\begin{frame}
As\'i, el t\'ermino $(\vec{\nabla}\sqrt{\rho})^{2}$ se escribir\'a como
\begin{equation}
\begin{array}{ll}
\left[\vec{\nabla}\sqrt{\rho (\vec{r})}\right]^{2} &=
\frac{1}{4\rho(\vec{r})}\left[\sum_{i} m_{i}\vec{\nabla}\delta(\vec{r}-\vec{r}_{i})\right]^{2}, \\\\\\ 
&=
\frac{4}{\lambda^{4}\rho(\vec{r})}
\left[
\sum_{i} m_{i}\delta(\vec{r}-\vec{r}_{i})(\vec{r}-\vec{r}_{i})
\right]^{2}. \label{eqn 3.15}

\end{array}
\end{equation}
y la densidad 
\begin{equation}
\rho(\vec{r})
=
\sum_{j}\sum_{i}m_{i}\delta(\vec{r}-\vec{r}_{j}),\label{eqn 3.16}
\end{equation} 
con $j$ denotando cada agrupaci\'on de c\'umulos.
\end{frame}
%%%%%%%%%%%%%%%%%%%%%%%%%%%%%%%%%%%%%%%%%%%%%%%%%%%%%%%%%%%%%%%%%%%%%%%%%%%%%%%%%%%%%%%%%%%%%%%%%%%%%%%%%%%%%%%%%%%%%%%%%
\begin{frame}
As\'i, el t\'ermino de $\vec{nabla}\sqrt{\rho}$ tiene la forma
\begin{equation}
\left[\vec{\nabla}\sqrt{\rho (\vec{r})}\right]^{2} \simeq 
\frac{4}{\lambda^{4}}
\left[
\sum_{i}m_{i}\delta(\vec{r}-\vec{r}_{j})(\vec{r}-\vec{r}_{j})
\right]^{2}
\left[
\sum_{j}m_{j}\delta(\vec{r}-\vec{r}_{j})
\right]^{-1}. \label{eqn 3.17}
\end{equation}
Integrando la ecuaci\'on (\ref{eqn 3.10}) se tiene
\begin{equation}
\int\left[\vec{\nabla}\sqrt{\rho (\vec{r})}\right]^{2}  \simeq 
\int \footnotesize{\frac{4 d^{3}x}{\lambda^{4}}
\left[
\sum_{j} m_{j}\delta(\vec{r}-\vec{r}_{j})(\vec{r}-\vec{r}_{j})
\right]^{2}
\left[
\sum_{j}m_{j}\delta(\vec{r}-\vec{r}_{j})
\right]^{-1}}\label{eqn 3.18}
\end{equation}
%\begin{equation}
%\simeq
%\footnotesize{4\lambda ^{-4} 
%\sum_{j}m_{j}\delta(\vec{r}-\vec{r}_{j})(\vec{r}-\vec{r}_{j})^{2}\Delta V_{j}%B_{j},}\label{eqn 3.19}
%\end{equation}
\begin{equation}
\simeq
\footnotesize{4\lambda^{-4}
\sum_{j}m_{j}
\frac{\Delta V_{j}B_{j}}{\lambda^{3}\pi^{3/2}}
\exp\left[-\frac{(\vec{r}-\vec{r}_{j})^{2}}{\lambda^{2}}\right]
(\vec{r}-\vec{r}_{j})^{2},} \label{eqn 3.20}
\end{equation}
donde $V_{j}$ y $B_{j}$ son el volumen efectivo y el factor de correcci\'on de la $j$-\'esima part\'icula. 
\end{frame}
%%%%%%%%%%%%%%%%%%%%%%%%%%%%%%%%%%%%%%%%%%%%%%%%%%%%%%%%%%%%%%%%%%%%%%%%%%%%%%%%%%%%%%%%%%%%%%%%%%%%%%%%%%%%%%%%%%%%%%%%
\begin{frame}
Finalmente, la ecuaci\'on (\ref{eqn 3.10}) se reacomoda
\begin{equation}
\footnotesize{\sum_{j} \frac{\partial K_{p}}{\partial q_{j}}
=
\frac{4\hbar^{2}}{m_{\chi}^{2}\lambda^{4}}
\sum_{j}m_{j}\Delta V_{j}B_{j}
\exp\left[
-\frac{2|\vec{r}-\vec{r}_{j}|^{2}}{\lambda^{2}}\right]
(1-\frac{2|\vec{r}-\vec{r}_{j}|^{2}}{\lambda^{2}})
(\vec{r}-\vec{r}_{j})}\label{eqn 3.21}
\end{equation}
al igual que la ecuaci\'on de movimiento (\ref{eqn 3.12})
\begin{equation}
\footnotesize{\sum_{j}m_{j}\ddot{q}_{j}
=
\frac{4\hbar^{2}}{m_{\chi}^{2}\lambda^{4}}
m_{j}\Delta V_{j}B_{j}
\exp\left[
-\frac{2|\vec{r}-\vec{r}_{j}|^{2}}{\lambda^{2}}\right]
(1-\frac{2|\vec{r}-\vec{r}_{j}|^{2}}{\lambda^{2}})
(\vec{r}-\vec{r}_{j}).}\label{eqn 3.22}
\end{equation}
Sustituyendo $q$ con $\vec{r}$, la aceleraci\'on adicional de la presi\'on cu\'antica en la simulaci\'on se describe como
\begin{equation}
\footnotesize{\ddot{\vec{r}}
=
\frac{4M\hbar^{2}}{M_{0}m_{\chi}^{2}\lambda^{4}}
\sum_{j}B_{j}
\exp\left[
-\frac{2|\vec{r}-\vec{r}_{j}|^{2}}{\lambda^{2}}\right]
(1-\frac{2|\vec{r}-\vec{r}_{j}|^{2}}{\lambda^{2}})
(\vec{r}_{j}-\vec{r}).}\label{eqn 3.23}
\end{equation}
La presi\'on cu\'antica es una interacci\'on de rango corto y es atractiva (repulsiva) si la distancia entre part\'iculas es mayor (menor) a $\lambda/\sqrt{2}$.

\end{frame}
%%%%%%%%%%%%%%%%%%%%%%%%%%%%%%%%%%%%%%%%%%%%%%%%%%%%%%%%%%%%%%%%%%%%%%%%%%%%%%%%%%%%%%%%%%%%%%%%%%%%%%%%%%%%%%%%%%%%%%%%
%%%%%%%%%%%%%%%%%%%%%%%%%%%%%%%%%%%%%%%%%%%%%%%%%%%%%%%%%%%%%%%%%%%%%%%%%%%%%%%%%%%%%%%%%%%%%%%%%%%%%%%%%%%%%%%%%%%%%%%%
%\begin{frame}
%\includemedia[width=0.6\linewidth,height=0.6\linewidth,activate=pageopen,
%passcontext,
%transparent,
%addresource=/home/jazhiel/Tesis_Jazhiel/Video/LCDM2M.mp4,
%flashvars={source=/home/jazhiel/Tesis_Jazhiel/Video/LCDM2M.mp4}
%]{\includegraphics[width=0.6\linewidth]{./Figuras/xy_00081_image}}{VPlayer.swf}
%\end{frame}
%%%%%%%%%%%%%%%%%%%%%%%%%%%%%%%%%%%%%%%%%%%%%%%%%%%%%%%%%%%%%%%%%%%%%%%%%%%%%%%%%%%%%%%%%%%%%%%%%%%%%%%%%%%%%%%%%%%%%%%%%
%%%%%%%%%%%%%%%%%%%%%%%%%%%%%%%%%%%%%%%%%%%%%%%%%%%%%%%%%%%%%%%%%%%%%%%%%%%%%%%%%%%%%%%%%%%%%%%%%%%%%%%%%%%%%%%%%%%%%%%%
\begin{frame}
\frametitle{Datos observacionales}
\centering
\includegraphics[width=0.5\textwidth]{./Figuras/cmb1}
\includegraphics[width=0.5\textwidth]{./Figuras/BBCosmology}
\includegraphics[width=0.4\textwidth]{./Figuras/materia-energia}
\end{frame}%%%%%%%%%%%%%%%%%%%%%%%%%%%%%%%%%%%%%%%%%%%%%%%%%%%%%%%%%%%%%%%%%%%%%%%%%%%%%%%%%%%%%%%%%%%%%%%%%%%%%%%%%%%%
\begin{frame}
\frametitle{Resultados}
\begin{table}[htb]
\centering
\begin{tabular}{|l|l|l|l|}
\hline
& \multicolumn{3}{c|}{Condiciones Iniciales} \\
\cline{2-4}
& Descripci\'on & S\'imbolo & Valor\\
\hline \hline
\multirow{3}{2cm}{Densidades \\ ($z=0$)}
& Materia oscura & $\Omega$ & 0.268\\ \cline{2-4}
& Constante cosmol\'ogica & $\Omega_{\Lambda}$ & 0.683\\ \cline{2-4}
& Materia Bari\'onica &  $\Omega_{b}$ & 0.049\\ \cline{1-4}
\hline
Tama\~no de simulaci\'on & Boxsize & $L$ & 5 Mpc\\ \cline{1-4}
\hline
N\'umero de part\'iculas & Cantidad & $N_{part}$ &$2^{21}\sim$ 2M\\ \cline{1-4}
\multirow{3}{1cm}{Tiempo} & Redshift inicial & $z_{i}$ & 15\\ \cline{2-4}
& Redshift final & $z_{f}$ & 0\\ \cline{1-4}
\hline
\multirow{3}{3cm}{Otros par\'ametros} & Par\'ametro de Hubble & $h$  & 0.7\\         \cline{2-4}
& espectro de potencias & $\sigma_{8}$ & 0.8\\ \cline{1-4}
%&Suavizamiento & $\epsilon$ & 1.81 kpc\\ \cline{1-4}
\end{tabular}
\caption{Condiciones creadas con N-GenIC.}
%\label{tabla:final}
\end{table}
\end{frame}
%%%%%%%%%%%%%%%%%%%%%%%%%%%%%%%%%%%%%%%%%%%%%%%%%%%%%%%%%%%%%%%%%%%%%%%%%%%%%%%%%%%%%%%%%%%%%%%%%%%%%%%%%%%%%%%%%%%%%%%%%%%%%
\begin{frame}
\begin{table}[htb]
\centering
\begin{tabular}{|l|l|l|l|}
\hline
& \multicolumn{3}{c|}{Par\'ametros F\'isicos} \\
\cline{2-4}
& Descripci\'on & Cantidad & Unidad\\
\hline \hline
\multirow{3}{2cm}{Sistema de Unidades}
& Longitud (cm) &  $3.08\times 10^{21}$  & 1 kpc\\ \cline{2-4}
& Masa (g) & $1.989\times 10^{43}$ & $10^{10}$ $M_{\odot}$\\ \cline{2-4}
& Velocidad (cm/s) & $10^{5}$  &1 km/s\\ \cline{1-4}
\multirow{3}{1cm}{Suavizados} & $\Lambda$CDM & $\epsilon$ & [0.5,1.81] kpc\\ \cline{2-4}
& SFDM & $\epsilon$ & [0.5,1.81] kpc\\ \cline{1-4}
\multirow{3}{2cm}{SFDM} & FdmAxionMass & $M_{\chi}$  & $10^{-22}$ eV \\ \cline{2-4}
& FdmKernelLambda & $\lambda$ & 1.14213562 kpc \\ \cline{1-4}
\end{tabular}
\caption{Cantidades adicionales para las simulaciones.}
%\label{tabla:final}
\end{table}
\end{frame}
%%%%%%%%%%%%%%%%%%%%%%%%%%%%%%%%%%%%%%%%%%%%%%%%%%%%%%%%%%%%%%%%%%%%%%%%%%%%%%%%%%%%%%%%%%%%%%%%%%%%%%%%%%%%%%
\begin{frame}
\begin{figure}
\centering
\includegraphics[width=1.0\textwidth]{./Figuras/xy_00081_image}
\end{figure}
\end{frame}
%%%%%%%%%%%%%%%%%%%%%%%%%%%%%%%%%%%%%%%%%%%%%%%%%%%%%%%%%%%%%%%%%%%%%%%%%%%%%%%%%%%%%%%%%%%%%%%%%%%%%%%%%%%%%%%%%%%%%%%%%%%%%%%%%%%%%%%%%%%%%%%%%%%%%
\begin{frame}
\begin{figure}
\includegraphics[width=0.8\textwidth]{./Figuras/masses_LCDM2M}
\caption{\footnotesize{Perfil de masas de halos de materia oscura, simulaci\'on de $\Lambda$CDM}}
\end{figure}
\end{frame}
%%%%%%%%%%%%%%%%%%%%%%%%%%%%%%%%%%%%%%%%%%%%%%%%%%%%%%%%%%%%%%%%%%%%%%%%%%%%%%%%%%%%%%%%%%%%%%%%%%%%%%%%%%%%%%%%%%%%%%%%%%%%%%%
\begin{frame}
\frametitle{zoom a 5 Mpc}
\centering
$\Lambda$CDM
\includegraphics[width=0.6\textwidth]{./Figuras/LCDM_093}\\
SFDM
\centering
\includegraphics[width=0.6\textwidth]{./Figuras/sfdm_10_22_eV}\\ $m\sim 10^{-22}eV$
\end{frame}
%%%%%%%%%%%%%%%%%%%%%%%%%%%%%%%%%%%%%%%%%%%%%%%%%%%%%%%%%%%%%%%%%%%%%%%%%%%%%%%%%%%%%%%%%%%%%%%%%%%%%
\begin{frame}
\frametitle{Espectros de masas de halos de materia oscura}
\begin{figure}
\includegraphics[width=0.45\textwidth]{./Figuras/masses_LCDM_1_81}
\includegraphics[width=0.45\textwidth]{./Figuras/masses_SFDM_1_81}
\end{figure}
\centering
$\Lambda$CDM \;\;\;\;\;\;\;\;\;\;\;\;\;\;\;\; SFDM
\end{frame}%%%%%%%%%%%%%%%%%%%%%%%%%%%%%%%%%%%%%%%%%%%%%%%%%%%%%%%%%%%%%%%%%%%%%%%%%%%%%%%%%%%%%%%%%%%%%%%%%%%%%%%
\begin{frame}
\centering
$\Lambda$CDM
\includegraphics[width=0.6\textwidth]{./Figuras/LCDM_093}\\
SFDM
\centering
\includegraphics[width=0.6\textwidth]{./Figuras/SFDM_10_23_eV}\\ $m\sim 10^{-23}eV$
\end{frame}%%%%%%%%%%%%%%%%%%%%%%%%%%%%%%%%%%%%%%%%%%%%%%%%%%%%%%%%%%%%%%%%%%%%%%%%%%%%%%%%%%%%%%%%%%%%%%%%
\begin{frame}
\begin{figure}
\includegraphics[width=0.45\textwidth]{./Figuras/masses_LCDM_1_81}
\includegraphics[width=0.45\textwidth]{./Figuras/masses_SFDM_1_81_10_23_eV}
\end{figure}
\centering
$\Lambda$CDM \;\;\;\;\;\;\;\;\;\;\;\;\;\;\;\; SFDM
\end{frame}
\begin{frame}
Diferencia de im\'agenes \\
\centering

$\Lambda$CDM
\includegraphics[width=0.6\textwidth]{./Figuras/LCDM_093}\\
SFDM
\includegraphics[width=0.6\textwidth]{./Figuras/SFDM}\\

$m\sim 10^{-23}eV$
\end{frame}
%%%%%%%%%%%%%%%%%%%%%%%%%%%%%%%%%%%%%%%%%%%%%%%%%%%%%%%%%%%%%%%%%%%%%%%%%%%%%%%%%%%%%%%%%%%%%%%%%%%%%%%%%%%%%%%%%%%%%%%%%%%%%%%%
\begin{frame}
\frametitle{Conclusiones y perspectivas}
\begin{itemize}
\setlength\itemsep{2em}
 \item Simulaciones num\'ericas permiten visualizar grandes sectores espaciales adem\'as de dar constricciones a condiciones iniciales.
 \item $\Lambda$CDM a\'un tiene problemas en escalas gal\'acticas.
 \item Teor\'ia lineal de perturbaciones para campo escalar aproxima de buena manera el colapso gravitacional de la materia oscura.
 \item SFDM es una buena alternativa para resolver ciertos problemas de CDM.
\end{itemize}
\end{frame}
%%%%%%%%%%%%%%%%%%%%%%%%%%%%%%%%%%%%%%%%%%%%%%%%%%%%%%%%%%%%%%%%%%%%%%%%%%%%%%%%%%%%%%%%%%%%%%%%%%%%%%%%%%%%%%%%
\begin{frame}
\begin{itemize}
\setlength\itemsep{2em}
\item Crear condiciones iniciales mejor adaptadas a observaciones (CAMB, CosmoMC, etc.)
\item Utilizar clusters computacionales con mayor procesamiento (Ya instalado en ABACUS)
\item Emplear teor\'ia de perturbaciones no lineal a segundo orden (2LPT) para simulaciones con $z>100$
\item Revisar y modificar el c\'odigo en busca de posibles errores.
\item Utilizar otros m\'etodos para simular $N$-cuerpos (RAMSES, \textcolor{blue}{AX-GADGET} , CONCEPT, etc.)
\item Entrar a la maestr\'ia
\end{itemize}
\end{frame}%%%%%%%%%%%%%%%%%%%%%%%%%%%%%%%%%%%%%%%%%%%%%%%%%%%%%%%%%%%%%%%%%%%%%%%%%%%%%%%%%%%%%%%%%%%%%%%
\begin{frame}
\centering
\Huge ¡GRACIAS!\\
\Huge :)
\end{frame}
%\begin{frame}
%\begin{figure}[htpb]
%\centering
%\subfigure[]{\includegraphics[width=0.3\textwidth]{./Figuras/movie_xy_00000}}
%\subfigure[]{\includegraphics[width=0.3\textwidth]{./Figuras/movie_xy_00030}}
%\subfigure[]{\includegraphics[width=0.3\textwidth]{./Figuras/movie_xy_00101}}
%\caption{\footnotesize{Evoluci\'on de dos galaxias espirales colisionando para %formar una sola. En la figura las partículas azules representan el halo de materia %oscura, las partículas rojas representan el disco estelar.}} \label{fig 3.3}
%\end{figure}
%\end{frame}
\end{document}