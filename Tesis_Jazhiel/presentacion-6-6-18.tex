\documentclass[10pt]{beamer}
\usepackage{graphicx}
\usepackage{subfigure}
\usepackage{media9}
\usepackage{multicol}
\usepackage{multirow} % para las tablas
\usetheme{Berlin}
%\usetheme[realshadow,corners=2pt,padding=2pt]{chamfered}
%\usecolortheme{shark}

\usepackage{tikz}
\newcommand<>{\hover}[1]{\uncover#2{%
 \begin{tikzpicture}[remember picture,overlay]%
 \draw[fill,opacity=0.4] (current page.south west)
 rectangle (current page.north east);
 \node at (current page.center) {#1};
 \end{tikzpicture}}
}

\title{Modelos de Materia Oscura: Una Perspectiva Num\'erica}
\author{Jazhiel Chac\'on Lavanderos}
\institute{Escuela Superior de F\'isica y Matem\'aticas, I.P.N.}
\date{\today}

\begin{document}

\begin{frame}
\maketitle
\end{frame}

\begin{frame}
\frametitle{Contenido}
\begin{itemize}
\item Introducci\'on
\item Simulaciones de $N$-Cuerpos
\item El C\'odigo
\item Resultados
\item Conclusiones y Perspectivas
\end{itemize}
\end{frame}

%%%%%%%%%%%%%%%%%%%%%%%%%%%%%%%%%%%%%%%%%%%%%%%%%%%%%%%%%%%%%%%%%%%%%%%%%%%%%%%%%%%%%%%%%%%%%%%%%%%%%%%%%%%%%%%%%%%%%%%%
%\begin{frame}
% \frametitle{Resumen}
%La naturaleza de la materia oscura ha sido y sigue siendo desconocida, f\'isicos te\'oricos y experimentales han trabajado en conjunto para poder descubrir su composici\'on durante las \'ultimas d\'ecadas. Una de las alternativas es crear simulaciones basadas en teor\'ias ya establecidas para poder explicar la existencia de la materia oscura y su composici\'on.
%\end{frame}
%%%%%%%%%%%%%%%%%%%%%%%%%%%%%%%%%%%%%%%%%%%%%%%%%%%%%%%%%%%%%%%%%%%%%%%%%%%%%%%%%%%%%%%%%%%%%%%%%%%%%%%%%%%%%%%%%%%%%%%%
\begin{frame}
\frametitle{Resumen}
En este trabajo, se abordan dos modelos de materia, uno de ellos conocido como el modelo est\'andar de la cosmolog\'ia y otro, utilizando la teor\'ia del campo escalar y su part\'icula elemental, un bos\'on ultra-ligero como componente de la materia oscura, recurriendo a un c\'odigo computacional que utiliza la teor\'ia de $N$-Cuerpos para simular la evoluci\'on cosmol\'ogica del Universo.

\end{frame}

%%%%%%%%%%%%%%%%%%%%%%%%%%%%%%%%%%%%%%%%%%%%%%%%%%%%%%%%%%%%%%%%%%%%%%%%%%%%%%%%%%%%%%%%%%%%%%%%%%%%%%%%%%%%%%%%%%%%%%%%
%\begin{frame}
%\frametitle{Introducci\'on a la Cosmolog\'ia}
%La cosmolog\'ia es la ciencia que estudia el Universo como un todo. El principio cosmol\'ogico es la base de cualquier modelo de evoluci\'on. Este principio aclara que \textbf{el Universo es homog\'eneo e is\'otropo en un espacio tridimensional.}
%\end{frame}
%%%%%%%%%%%%%%%%%%%%%%%%%%%%%%%%%%%%%%%%%%%%%%%%%%%%%%%%%%%%%%%%%%%%%%%%%%%%%%%%%%%%%%%%%%%%%%%%%%%%%%%%%%%%%%%%%%%%%%%%

%Basado en la Teor\'ia del Big Bang y en la Relatividad, la intuici\'on indicar\'ia que si, el Universo se expande, pero que en alg\'un punto en el tiempo, esta expansi\'on deber\'ia reducirse y detenerse, toda la materia tendr\'ia que recolapsar sobre si misma, debido a la autogravitaci\'on.

%%%%%%%%%%%%%%%%%%%%%%%%%%%%%%%%%%%%%%%%%%%%%%%%%%%%%%%%%%%%%%%%%%%%%%%%%%%%%%%%%%%%%%%%%%%%%%%%%%%%%%%%%%%%%%%%%%%%%%%%

\begin{frame}
\frametitle{Introducci\'on}
En 1927, estudiando la din\'amica de nebulosas extragal\'acticas, Edwin Hubble concluye que las galaxias en el Universo est\'an en expansi\'on, a raz\'on
\begin{equation}
\vec{v} = H \vec{r},\label{eqn 1}
\end{equation}
donde $H(t)$ es el \textit{par\'ametro de Hubble}, en coordenadas com\'oviles
\begin{equation}
\vec{r}=a(t)\vec{x},\label{eqn 2}
\end{equation}
donde $a(t)$ es el \textit{factor de escala} del Universo. Entonces
\begin{equation}
\vec{v}=\vec{\dot{r}} = H\vec{r},\label{eqn 3}
\end{equation}
\begin{equation}
\frac{d}{d t}(a\vec{x}) = Ha\vec{x},\label{eqn 4}
\end{equation}
entonces
\begin{equation}
H(t) = \frac{\dot{a}(t)}{a(t)}\label{eqn 5} 
\end{equation}


\end{frame}
%%%%%%%%%%%%%%%%%%%%%%%%%%%%%%%%%%%%%%%%%%%%%%%%%%%%%%%%%%%%%%%%%%%%%%%%%%%%%%%%%%%%%%%%%%%%%%%%%%%%%%%%%%%%%%%%%%%%%%%%

%\begin{frame}
%La luz de las galaxias tiene diferentes caracter\'isticas espectrales propias de los \'atomos que la componen, al examinar l\'ineas de emisi\'on se concluye que se desplazan al rojo del especto. Este cambio indica que todas las galaxias se alejan y se define como 
%\begin{equation}
%z \equiv \frac{\lambda-\lambda_{0}}{\lambda_{0}},\label{eqn 6}
%\end{equation}
%la velocidad relativa entre galaxias viene dada por 
%\begin{equation}
%d\vec{v} = H d\vec{r} = \frac{\dot{a}}{a}cdt = c \frac{da}{a}\label{eqn 7}
%\end{equation}
%donde $c$ es la velocidad de la luz de un fot\'on emitido por una galaxia. Si se escribe  el cambio en la longitud de onda emitida como $d\lambda = \lambda - \lambda_{0}$\label{eqn 8}, sustituyendo en la ecuaci\'on (\ref{eqn 7}) se obtiene 
%\begin{equation}
%c\frac{da}{a} = d \vec{v} = cz = c\frac{d\lambda}{\lambda}.%\label{eqn 8}
%\end{equation}
%\end{frame}
%%%%%%%%%%%%%%%%%%%%%%%%%%%%%%%%%%%%%%%%%%%%%%%%%%%%%%%%%%%%%%%%%%%%%%%%%%%%%%%%%%%%%%%%%%%%%%%%%%%%%%%%%%%%%%%%%%%%%%%%
\begin{frame}
\frametitle{Ecuaciones de Einstein}
Indican la curvatura del espacio-tiempo del Universo debido a la presencia de materia-energ\'ia en el mismo, estas son 10 ecuaciones diferenciales acopladas y rigen los sistemas gravitacionales. Introduciendo un espacio-tiempo mediante la ecuaci\'on de la m\'etrica
\begin{equation}
ds^{2} = g_{\alpha\beta}dx^{\alpha}dx^{\beta},\label{eqn 9}
\end{equation}
se obtiene
\begin{equation}
R_{\alpha\beta} - \frac{1}{2} R g_{\alpha\beta} + \Lambda g_{\alpha\beta} = 0,\label{eqn 10}
\end{equation}
donde $R{\alpha\beta}$ es el tensor de Riemann, $R$ el escalar de curvatura de Ricci, $g_{\alpha\beta}$ es el tensor m\'etrico y $\Lambda$ es la constante cosmol\'ogica. En presencia de materia y energ\'ia, estas ecuaciones de campo de Einstein toman la siguiente forma
\begin{equation}
R_{\alpha\beta} - \frac{1}{2} R g_{\alpha\beta} + \Lambda g_{\alpha\beta} = \kappa T_{\alpha\beta}, \label{eqn 11}
\end{equation}
donde $T_{\alpha\beta}$ es el tensor de energ\'ia-momento y $\kappa = 8\pi G/c^{4}$.
\end{frame}
%%%%%%%%%%%%%%%%%%%%%%%%%%%%%%%%%%%%%%%%%%%%%%%%%%%%%%%%%%%%%%%%%%%%%%%%%%%%%%%%%%%%%%%%%%%%%%%%%%%%%%%%%%%%%%%%%%%%%%%%
\begin{frame}
\frametitle{M\'etrica de Friedmann-Lema\^itre-Robertson-Walker}
Dados dos eventos en el espacio--tiempo, uno ocurriendo en el punto localizado en $(t,r,\theta , \phi)$ y otro ocurriendo en el punto localizado en $(t +dt, r + dr, \theta + d\theta, \phi + d\phi)$. La separaci\'on espacio-temporal entre estos dos eventos es
\begin{equation}
ds^{2}=-c^{2}dt^{2} + dr_{2} +r^{2}d\Omega^{2},\label{eqn 12}
\end{equation}
donde
\begin{equation*}
d\Omega^{2} \equiv d\theta^{2} + \sin^{2}\theta d\phi^{2}.
\end{equation*}
La m\'etrica descrita por la ecuaci\'on (\ref{eqn 12}) es llamada \textit{m\'etrica de Minkowski}. Esta m\'etrica solo aplica dentro de la relatividad especial, el resultado es una m\'etrica de un espacio--tiempo plano.
\end{frame}
%%%%%%%%%%%%%%%%%%%%%%%%%%%%%%%%%%%%%%%%%%%%%%%%%%%%%%%%%%%%%%%%%%%%%%%%%%%%%%%%%%%%%%%%%%%%%%%%%%%%%%%%%%%%%%%%%%%%%%%%
\begin{frame}
%\frametitle{M\'etrica de Friedmann-Lema\^itre-Robertson-Walker}
Al a\~nadir gravedad, la m\'etrica cambia. En la d\'ecada de 1930 los f\'isicos Howard Robertson y Arthur Walker llegaron a un resultado, de manera independiente, que se conoci\'o como \textit{m\'etrica de Robertson--Walker} y cuya forma es generalmente conocida como
\begin{equation}
ds^{2}
=
-c^{2}dt^{2} + a^{2}(t)
\left[
\frac{dr^{2}}{1-k r^{2}} + r^{2}d\Omega^{2}
\right],\label{eqn 13}
\end{equation}
donde $k > 0$, $k = 0$ o $k < 0$ describen espacios abiertos, planos o cerrados, respectivamente.
\begin{figure}
\centering
    \includegraphics[width=0.4\textwidth]{./Figuras/ScaleFactor}
  \caption{\footnotesize{El factor de escala cosmológico $a(t)$ para modelos de Universo abierto $(k = -1)$, plano $(k = 0)$ y cerrado $(k = 1)$.}}
  \label{fig 1}
\end{figure}
\end{frame}
%%%%%%%%%%%%%%%%%%%%%%%%%%%%%%%%%%%%%%%%%%%%%%%%%%%%%%%%%%%%%%%%%%%%%%%%%%%%%%%%%%%%%%%%%%%%%%%%%%%%%%%%%%%%%%%%%%%%%%%%
\begin{frame}
Para resolver las ecuaciones de campo de Einstein bajo el principio cosmol\'ogico, se necesita un tensor de energ\'ia-momento igualmente homog\'eneo e isotr\'opico. Este es el de un fluido perfecto, que se escribe como
\begin{equation}
T_{\mu \nu} =
(\rho + p)u_{\mu}u_{\nu} + pg_{\mu \nu},\label{eqn 14}
\end{equation}
donde $\rho$ es la densidad de masa propia del fluido, $u_{\mu}$ es la cuadrivelocidad y $p$ es la presi\'on $(p > 0)$ o la tensi\'on $(p < 0)$. Para la m\'etrica de la ecuaci\'on (\ref{eqn 13}) este tensor es diagonal
\begin{equation}
T_{\mu \nu} = \textup{diag}(\rho, p, p, p).\label{eqn 15}
\end{equation}
\end{frame}
%%%%%%%%%%%%%%%%%%%%%%%%%%%%%%%%%%%%%%%%%%%%%%%%%%%%%%%%%%%%%%%%%%%%%%%%%%%%%%%%%%%%%%%%%%%%%%%%%%%%%%%%%%%%%%%%%%%%%%%%
\begin{frame}
La expansi\'on m\'etrica del espacio se describe con las ecuaciones de Friedmann, que se encuentran a partir de las ecuaciones de campo de Einstein para la m\'etrica FLRW. Recordando la relaci\'on del par\'ametro de Hubble $H(t)=\dot{a}/a$ e insertando la ecuaci\'on (\ref{eqn 15}) en (\ref{eqn 11}) y tomando $\Lambda = 0$ y $c=1$
\begin{equation*}
 3\frac{\dot{a}^{2} + k}{a^{2}}  = 8 \pi G \rho
 \end{equation*}
\begin{equation} 
-2\frac{\ddot{a}}{a} - \frac{\dot{a}^{2} + k}{a^{2}}= 8 \pi G p.\label{eqn 16}
\end{equation}
Al combinar ambas ecuaciones se obtiene
\begin{equation}
\frac{\ddot{a}}{a} = -\frac{4 \pi G}{3}(\rho + 3p), \label{eqn 17}
\end{equation}
Las ecuaciones de Friedmann describen la cantidad de materia contenida en el Universo y determinan su geometr\'ia.
\end{frame}
%%%%%%%%%%%%%%%%%%%%%%%%%%%%%%%%%%%%%%%%%%%%%%%%%%%%%%%%%%%%%%%%%%%%%%%%%%%%%%%%%%%%%%%%%%%%%%%%%%%%%%%%%%%%%%%%%%%%%%%%
%\begin{frame}
%\frametitle{$\Lambda$CDM}
%Aunque la primera evidencia de materia oscura fue descubierta en la d\'ecada de 1930, no fue hasta la d\'ecada de 1980 que los astr\'onomos se convencieron de que \'esta es el componente responsable que mantiene unidas a las galaxias y los c\'umulos de galaxias de manera gravitacional.

%El modelo \textit{Lambda Cold Dark Matter} ($\Lambda$CDM) es una parametrizaci\'on del modelo del Big Bang. Tambi\'en conocido como el ``modelo est\'andar de la cosmolog\'ia'' se  fundamenta,  principalmente, sobre las siguientes bases te\'oricas y experimentales.
%\end{frame}
%%%%%%%%%%%%%%%%%%%%%%%%%%%%%%%%%%%%%%%%%%%%%%%%%%%%%%%%%%%%%%%%%%%%%%%%%%%%%%%%%%%%%%%%%%%%%%%%%%%%%%%%%%%%%%%%%%%%%%%%
\begin{frame}
\begin{figure}
\centering
\includegraphics[width=0.6\textwidth]{./Figuras/materia-energia}
\caption{\footnotesize{Contenido estimado del Universo.}}
\end{figure}
\end{frame}
%%%%%%%%%%%%%%%%%%%%%%%%%%%%%%%%%%%%%%%%%%%%%%%%%%%%%%%%%%%%%%%%%%%%%%%%%%%%%%%%%%%%%%%%%%%%%%%%%%%%%%%%%%%%%%%%%%%%%%%%%%%%%%%%%%%%%%%%%%%%%%%%%%%%%%%%%%%%%%
\begin{frame}
\begin{itemize}
\frametitle{$\Lambda$CDM}
\item Un marco te\'orico basado en la teor\'ia general de la relatividad, que proporciona la teor\'ia del campo gravitatorio en escalas cosmol\'ogicas.
\item El principio cosmol\'ogico.
\item El modelo de fluidos, que considera a las galaxias como constituyentes b\'asicos del Universo.
\item La Ley de Hubble.
\item La Radiaci\'on del Fondo C\'osmico de Microondas.
%\item La determinaci\'on de la abundancia relativa de elementos primigenios $^{1}$H, $^{2}$D, $^{3}$He, $^{4}$He y $^{7}$Li formados en las reacciones nucleares en la \'epoca de Big Bang Nucleos\'intesis.
%\item El an\'alisis de la estructura a gran escala del Universo, mediante experimentos como el SDSS, que atestiguan la homogeneidad y ayudan a la determinaci\'on de los distintos par\'ametros del modelo est\'andar.
\end{itemize}
\end{frame}
%%%%%%%%%%%%%%%%%%%%%%%%%%%%%%%%%%%%%%%%%%%%%%%%%%%%%%%%%%%%%%%%%%%%%%%%%%%%%%%%%%%%%%%%%%%%%%%%%%%%%%%%%%%%%%%%%%%%%%%%
\begin{frame}
\begin{itemize}
\item Perturbaciones a la densidad. Tambi\'en conocidas como fluctuaciones de densidad o fluctuaciones cu\'anticas.

\item La \textit{Inflaci\'on}, una expansi\'on acelerada, propuesta originalmente por Alan Guth, y que explica la planitud y la homogeneidad actuales del Universo.

%\item El Hot Big Bang, origen extremadamente caliente que da lugar a BBN.

\item La \textit{constante cosmol\'ogica} $\Lambda$, que Einstein introdujo en las ecuaciones de la relatividad general
\begin{equation}
H^{2}(t) = \left(\frac{\dot{a}}{a}\right)^{2}
= 
\frac{8 \pi G}{3} - \frac{k}{a^{2}} + \frac{\Lambda}{3}.\label{eqn 18}
\end{equation}
\item La \textit{materia oscura fr\'ia}, Cold Dark Matter (CDM). Un tipo de materia que debe actuar de forma exclusivamente gravitatoria, que es oscura o \textit{transparente} (no interact\'ua con ning\'un tipo de materia bari\'onica o radiaci\'on) y que no debe moverse a velocidades relativistas (es fr\'ia).
\end{itemize}

\end{frame}
%%%%%%%%%%%%%%%%%%%%%%%%%%%%%%%%%%%%%%%%%%%%%%%%%%%%%%%%%%%%%%%%%%%%%%%%%%%%%%%%%%%%%%%%%%%%%%%%%%%%%%%%%%%%%%%%%%%%%%%%
\begin{frame}
\begin{figure}
\centering
 \subfigure[]{\includegraphics[width=0.4\textwidth]{./Figuras/planck}}
 \subfigure[]{\includegraphics[width=0.4\textwidth]{./Figuras/BBCosmology}}
  \caption{\footnotesize{Fondo C\'osmico de Microondas y l\'inea del tiempo del Universo.}}
  \label{fig 1.6}
\end{figure}
\end{frame}
%%%%%%%%%%%%%%%%%%%%%%%%%%%%%%%%%%%%%%%%%%%%%%%%%%%%%%%%%%%%%%%%%%%%%%%%%%%%%%%%%%%%%%%%%%%%%%%%%%%%%%%%%%%%%%%%%%%%%%%%
\begin{frame}
\frametitle{Problemas con $\Lambda$CDM}
\begin{itemize}
\item Problema CUSP-CORE
\item Sat\'elites faltantes
\item Problema de la concordancia
\end{itemize}
\end{frame}
%%%%%%%%%%%%%%%%%%%%%%%%%%%%%%%%%%%%%%%%%%%%%%%%%%%%%%%%%%%%%%%%%%%%%%%%%%%%%%%%%%%%%%%%%%%%%%%%%%%%%%%%%%%%%%%%%%%%%%%%SFDM
\begin{frame}
\frametitle{Scalar Field Dark Matter (SFDM)}
Este modelo supone que la materia oscura es un campo escalar real o complejo $\Phi$ m\'inimamente acoplado a la gravedad, dotado de un potencial escalar $V(\Phi)$ y que a cierta temperatura la interacci\'on  del campo es puramente gravitacional junto con el resto de la materia.
\end{frame}
%%%%%%%%%%%%%%%%%%%%%%%%%%%%%%%%%%%%%%%%%%%%%%%%%%%%%%%%%%%%%%%%%%%%%%%%%%%%%%%%%%%%%%%%%%%%%%%%%%%%%%%%%%%%%%%%%%%%%%%%
\begin{frame}
\frametitle{Campos escalares}
%%%PONER CAMPO ESCALAR TEORIA
%En el modelo de Campo Escalar, se propone que los halos gal\'acticos se forman de condensados de Bose-Einstein de un campo escalar (SF) cuyo bos\'on tiene una masa ultra ligera del orden de $m \sim 10^{-22}$eV. De este valor se sigue que la temperatura cr\'itica de condensaci\'on $T_{c} \sim 1/m^{5/3} \sim $ TeV, es muy alta, por lo tanto, se forman semillas de Condensados de Bose-Einstein (CBE) en \'epocas tempranas en el Universo.
\end{frame}
%%%%%%%%%%%%%%%%%%%%%%%%%%%%%%%%%%%%%%%%%%%%%%%%%%%%%%%%%%%%%%%%%%%%%%%%%%%%%%%%%%%%%%%%%%%%%%%%%%%%%%%%%%%%%%%%%%%%%%%%
\begin{frame}
Recordando las ecuaciones de FLRW, el tensor energ\'ia-momento \textbf{T} para un campo escalar, la densidad de energ\'ia escalar $T_{0}^{0}$ y la presi\'on escalar $T_{j}^{i}$ estar\'an dadas por
\begin{equation}
T_{0}^{0}=-\rho_{\Phi_{0}}=-\left(\frac{1}{2}\dot{\Phi}_{0}^{2} + V \right),\label{eqn 19}
\end{equation}
y
\begin{equation}
T_{j}^{i}=P_{\Phi_{0}}=\left(\frac{1}{2} \dot{\Phi}_{0}^{2}-V \right)\delta_{j}^{i},\label{eqn 20}
\end{equation}
donde el punto se entiende como la derivada respecto al tiempo cosmol\'ogico y $\delta_{j}^{i}$ es la delta de Kronecker.
\end{frame}
%%%%%%%%%%%%%%%%%%%%%%%%%%%%%%%%%%%%%%%%%%%%%%%%%%%%%%%%%%%%%%%%%%%%%%%%%%%%%%%%%%%%%%%%%%%%%%%%%%%%%%%%%%%%%%%%%%%%%%%%
%\begin{frame}
%la Ecuaci\'on de Estado para el campo escalar es $p_{\Phi_{0}}=\omega_{\Phi_{0}}\rho_{\Phi_{0}}$ con
%\begin{equation}
%\omega_{\Phi_{0}} = \frac{\frac{1}{2}\dot{\Phi}_{0}^{2}-V}{\frac{1}{2}\dot{\Phi}_{0}^{2}+V}.\label{eqn 21}
%\end{equation}
%Se definen nuevas variables adimensionales
%\begin{equation}
%x\equiv \frac{\kappa}{\sqrt{6}}\frac{\Phi_{0}}{H}, \;\;\; u\equiv\frac{\kappa}{\sqrt{3}}\frac{\sqrt{V}}{H}\label{eqn 22}
%\end{equation}
%donde $\kappa^{2}\equiv 8\pi G$ y $H \equiv \dot{a}/a$ es el par\'ametro de Hubble. Se toma el potencial escalar como $V = m^{2}\Phi^{2}/2\hbar^{2} + \lambda\Phi^{4}/4$, si se toma $c = 1$, para un bos\'on ultra ligero se tendr\'a que $\mu_{\Phi} \sim 10^{-22} $eV.
%\end{frame}
%%%%%%%%%%%%%%%%%%%%%%%%%%%%%%%%%%%%%%%%%%%%%%%%%%%%%%%%%%%%%%%%%%%%%%%%%%%%%%%%%%%%%%%%%%%%%%%%%%%%%%%%%%%%%%%%%%%%%%%%
%\begin{frame}
%\frametitle{Aproximaci\'on hidrodin\'amica}
%En esta aproximaci\'on, se hace una transformaci\'on para %resolver las ecuaciones de Friedmann de manera anal\'itica con %la condici\'on $H<<m$. 
%Se toma el potencial escalar como $V = m^{2}\Phi^{2}/2\hbar^{2} + \lambda\Phi^{4}/4$. As\'i, para el bos\'on ultra ligero se tiene que $m \sim 10^{-22}$ eV. $\Phi_{0}$ se expresa en t\'erminos de nuevas variables $S$ y $\rho_{0}$, donde $S$ es constante en el fondo y $\rho_{0}$ ser\'a la densidad de energ\'ia del fluido tambi\'en en esta regi\'on, así el campo en dicha región se expresa como
%\begin{equation}
%\Phi_{0} = (\psi_{0}e^{-imt/\hbar} + \psi_{0}^{*}e^{imt/\hbar}),\label{eqn 23}
%\end{equation}
%donde
%\begin{equation}
%\psi_{0}(t) = \sqrt{\rho_{0}(t)}e^{iS/\hbar},\label{eqn 24}
%\end{equation}
%As\'i, el campo escalar en la regi\'on del fondo del Universo se puede expresar como 
%\begin{equation}
%\Phi_{0}=2\sqrt{\rho_{0}}\cos(S-mt/\hbar),\label{eqn 25}
%\end{equation}
%con esto se obtiene
%\begin{equation}
%\dot{\Phi}_{0}^{2} = \rho_{0} 
%\left[
%\frac{\dot{\rho}_{0}}{\rho_{0}}\cos(S-mt/\hbar) 
%- 2(\dot{S}-mt/\hbar)\sin(S-mt/\hbar)
%\right]^{2}.\label{eqn 26}
%\end{equation}
%Observe que el principio de incertidumbre implica que $m \Delta t \sim \hbar$, y que para el fondo en el caso no relativista se cumple la relación $\dot{S}/m \sim 0$.
%\end{frame}
%%%%%%%%%%%%%%%%%%%%%%%%%%%%%%%%%%%%%%%%%%%%%%%%%%%%%%%%%%%%%%%%%%%%%%%%%%%%%%%%%%%%%%%%%%%%%%%%%%%%%%%%%%%%%%%%%%%%%%%%
\begin{frame}
\frametitle{Aproximaci\'on hidrodin\'amica y perturbaciones}
%El campo escalar tiene oscilaciones intensas desde el inicio, estas oscilaciones se transmiten a las fluctuaciones que crecen de manera r\'apida. 
Una perturbaci\'on en cualquier cantidad es la diferencia entre su valor correspondiente en un evento en el espacio--tiempo real y su correspondiente valor de ``fondo'', as\'i, para el campo escalar, se tiene
\begin{equation}
\Phi = \Phi_{0}(t) + \delta\Phi(\vec{x},t),\label{eqn 27}
\end{equation}
que al insertar en la ecuaci\'on de Klein-Gordon perturbada, se tiene, con $\dot{\phi}=0$
\begin{equation}
\delta\ddot{\Phi} + 3H\delta\dot{\Phi}
- \frac{1}{a^{2}}\vec{\nabla}^{2}\delta\Phi
+V_{,\Phi\Phi}\delta\Phi +2V_{,\Phi}\phi = 0.\label{eqn 28}
\end{equation}
Entonces el campo escalar perturbado $\delta\Phi$ en t\'erminos de $\Psi$ puede expresarse de la siguiente manera
\begin{equation}
\delta\Phi = \Psi e^{-imt/\hbar} +\Psi^{*}e^{imt/\hbar},\label{eqn 29}
\end{equation}
\end{frame}
%%%%%%%%%%%%%%%%%%%%%%%%%%%%%%%%%%%%%%%%%%%%%%%%%%%%%%%%%%%%%%%%%%%%%%%%%%%%%%%%%%%%%%%%%%%%%%%%%%%%%%%%%%%%%%%%%%%%%%%%%%%%%%%%%%%%%%%%%%%%%%%%%%%%%%%%%%%%%%%%%%%%%%%%%%%%%%%%%%%%%%%%%%%%%%%%%%%%%%%%%%%%%%%%%%%%%%%%%%%%%%%%%%%%%%%%%%%%%%%%
\begin{frame}
Con esta ecuaci\'on y la expresi\'on del potencial del campo escalar, la ecuaci\'on (\ref{eqn 28}) se convierte en 
\begin{equation}
-i\hbar(\dot{\Psi}+\frac{3}{2}H\Psi) + \frac{\hbar^{2}}{2m}(\Box \Psi +9\lambda|\Psi|^{2}\Psi) + m\phi\Psi = 0,\label{eqn 30}
\end{equation}
donde $\Box$ se define como 
\begin{equation}
\Box = \frac{d^{2}}{d t^{2}} + 3H\frac{d}{d t} - \frac{1}{a^{2}}\vec{\nabla}^{2}\label{eqn 31}
\end{equation}
Para entender la naturaleza hidrodin\'amica de este modelo, se hace una aproximaci\'on utilizando una transformaci\'on de Madelung, la cual conecta la teor\'ia de campos y las funciones de onda de los condensados
\end{frame}
%%%%%%%%%%%%%%%%%%%%%%%%%%%%%%%%%%%%%%%%%%%%%%%%%%%%%%%%%%%%%%%%%%%%%%%%%%%%%%%%%%%%%%%%%%%%%%%%%%%%%%%%%%%%%%%%%%%%%%%%
\begin{frame}
\begin{equation}
\Psi=\sqrt{\hat{\rho}} e^{iS},\label{eqn 32}
\end{equation}
$\Psi$ ser\'a la funci\'on de onda del condensado, con $\hat{\rho}=\rho/m=\hat{\rho}(\vec{x},t)$ y $S=S(\vec{x},t)$. la funci\'on $\Psi$ se separa en una fase real $S$ y una amplitud real $\hat{\rho}$, mientras que se satisface la condici\'on $|\Psi|^{2}=\Psi\Psi^{*}= \hat{\rho}$. sustituyendo en la ecuaci\'on (\ref{eqn 30}), se obtiene
\begin{equation}
\dot{\hat{\rho}} + 3H\hat{\rho}
-\frac{\hbar}{m}\hat{\rho}\Box S 
+\frac{\hbar}{a^{2}m}\vec{\nabla}S\vec{\nabla}\hat{\rho}
-\frac{\hbar}{m}\hat{\rho}\dot{S}=0,\label{eqn 33}
\end{equation}
y
\begin{equation}
\hbar \dot{S}/m + \omega\hat{\rho}
+ \phi
+ \frac{\hbar^{2}}{2m^{2}}\left(\frac{\Box\sqrt{\hat{\rho}}}{\sqrt{\hat{\rho}}}\right)
+ \frac{\hbar^{2}}{2a^{2}}[\vec{\nabla}(S/m)]^{2}
- \frac{\hbar^{2}}{2}(\dot{S}/m)^{2} = 0.\label{eqn 34}
\end{equation}
\end{frame}%%%%%%%%%%%%%%%%%%%%%%%%%%%%%%%%%%%%%%%%%%%%%%%%%%%%%%%%%%%%%%%%%%%%%%%%%%%%%%%%%%%%%%%%%%%%%%%%%%%%%%%%%%%
\begin{frame}
Tomando el gradiente de las ecuaciones (\ref{eqn 33}), (\ref{eqn 34}) dividiendo por $a$ y utilizando la definici\'on 
\begin{equation}
\vec{v}\equiv \frac{\hbar}{ma}\vec{\nabla}S\label{eqn 35}
\end{equation}
se obtiene
\begin{equation}
\dot{\hat{\rho}} + 3H\hat{\rho} - \frac{\hbar}{m}\hat{\rho}\Box S 
+ \frac{1}{a}\vec{v}\vec{\nabla}\hat{\rho} - \frac{\hbar}{m}\hat{\rho}\dot{S} = 0,\label{eqn 36}
\end{equation}
\begin{equation*}
\dot{\vec{v}} + H\vec{v} + \frac{1}{2a\hat{\rho}\vec{\nabla}}p 
+ \frac{1}{a}\vec{\nabla}\phi + \frac{\hbar^{2}}{2m^{2}a}\vec{\nabla}
\left(\frac{\Box\sqrt{\hat{\rho}}}{\sqrt{\hat{\rho}}}\right) 
\end{equation*}  
\begin{equation}
+\frac{1}{a}(\vec{v}\cdot\vec{\nabla})\vec{v}-\hbar(\dot{\vec{v}}+H\vec{v})(\dot{S}/m) = 0\label{eqn 37}
\end{equation}
donde se ha definido $\omega = 9\hbar^{2}\lambda/2m^{2}$ y $p= \omega\hat{ \rho}^{2}$. 
Al ignorar los t\'erminos cuadr\'aticos y derivadas temporales de segundo orden en las ecuaciones (\ref{eqn 36}) y (\ref{eqn 37}) se obtiene
\end{frame}
%%%%%%%%%%%%%%%%%%%%%%%%%%%%%%%%%%%%%%%%%%%%%%%%%%%%%%%%%%%%%%%%%%%%%%%%%%%%%%%%%%%%%%%%%%%%%%%%%%%%%%%%%%%
\begin{frame}
\begin{equation}
\frac{\partial \hat{\rho}}{\partial t} +
\vec{\nabla}\cdot(\hat{\rho} \vec{v}) + 3H\hat{\rho}=0,\label{eqn 38}
\end{equation}
\begin{equation}
\frac{\partial \vec{v}}{\partial t} + H \vec{v}
-\frac{\hbar^{2}}{2m^{2}}\vec{\nabla}\left(\frac{1}{2\hat{\rho}}\vec{\nabla}^{2}\hat{\rho}\right) + \omega\vec{\nabla}\hat{\rho} +\vec{\nabla}^{2}\phi,\label{eqn 39}
\end{equation}
\begin{equation}
\vec{\nabla}^{2}\phi = 4\pi G\hat{\rho},\label{eqn 40}
\end{equation}
Las ecuaciones (\ref{eqn 36}) y (\ref{eqn 37}) son el an\'alogo de las ecuaciones de Euler para fluidos cl\'asicos, con la diferencia de la existencia de un t\'ermino cu\'antico, llamado $Q$ y definido por
\begin{equation}
Q = \frac{\hbar^{2}}{2m^{2}}\frac{\Box\sqrt{\hat{\rho}}}{\sqrt{\hat{\rho}}},\label{eqn 1.78}
\end{equation}
este \'ultimo t\'ermino, puede describir una fuerza o ``presi\'on '' negativa de naturaleza cu\'antica.
\end{frame}
%%%%%%%%%%%%%%%%%%%%%%%%%%%%%%%%%%%%%%%%%%%%%%%%%%%%%%%%%%%%%%%%%%%%%%%%%%%%%%%%%%%%%%%%%%%%%%%%%%%%%%%%%%%%%%%%%%%%%%%%
\begin{frame}
\frametitle{Simulaciones de $N$-Cuerpos}
%La evoluci\'on de estructura se aproxima con aglomeramiento gravitacional  no lineal a partir de condiciones iniciales espec\'ificas de part\'iculas de materia oscura y puede refinarse introduciendo efectos de din\'amica de gases, procesos qu\'imicos, transferencia radiativa y otros procesos astrof\'isicos.
La evoluci\'on no lineal de la materia oscura utilizan la ecuaci\'on de Boltzmann no colisional en coordenadas com\'oviles acoplada con la ecuaci\'on de Poisson
\begin{equation}
\frac{\partial f}{\partial t} + \vec{v}\cdot\vec\nabla_{r} + \frac{\vec{F}}{m}\cdot\vec\nabla_{v}=0,\label{eqn 39}
\end{equation}
donde $f= f(\vec{r}, \vec{p}, t)$ es una funci\'on de distribuci\'on de la densidad, $\vec{v}$ es la velocidad, $\vec{r}$ es la posici\'on, $\vec{F}$ es la fuerza y $m$ la masa que describen completamente al fluido.

En el caso de que esta fuerza $\vec{F}$ se derive de un potencial, tal que 
\begin{equation}
\vec{F} = -m\nabla\Phi,\label{eqn 40}
\end{equation}
donde $m$ es la masa de la partícula del sistema. Sustituyendo (\ref{eqn 40}) en (\ref{eqn 39}) se encuentra
\begin{equation}
\frac{\partial f}{\partial t} + \vec{v}\cdot\vec\nabla_{r} - \vec\nabla\Phi\cdot\vec\nabla_{v}=0\label{eqn2.3}
\end{equation} 
%Este potencial $\Phi$ debe satisfacer la ecuación de Poisson
%\begin{equation}
%\nabla^{2} \Phi (\vec{r},t) = 4\pi \int_{S} f(\vec{r},\vec{v},t)d^{3}v\label{eqn2.4}
%\end{equation}
%donde $S$ es todo el espacio y $f$ se define mediante la siguiente expresión
%\begin{equation}
%f = f(\vec{r},\vec{v},t)d^{3}v d^{3}r\label{eqn 2.5}
%\end{equation}

\end{frame}
%%%%%%%%%%%%%%%%%%%%%%%%%%%%%%%%%%%%%%%%%%%%%%%%%%%%%%%%%%%%%%%%%%%%%%%%%%%%%%%%%%%%%%%%%%%%%%%%%%%%%%%%%%%%%%%%%%%%%%%%
%\begin{frame}
%\frametitle{Fluidos sin Colisi\'on Autogravitantes}
%\end{frame}
%%%%%%%%%%%%%%%%%%%%%%%%%%%%%%%%%%%%%%%%%%%%%%%%%%%%%%%%%%%%%%%%%%%%%%%%%%%%%%%%%%%%%%%%%%%%%%%%%%%%%%%%%%%%%%%%%%%%%%%%
\begin{frame}
El problema que se intenta resolver es el siguiente: dadas las coordenadas iniciales $\vec{r}_{init}$ y velocidades $\vec{v}_{init}$ de $N$ part\'iculas con masa en el momento $t = t_{init}$, encontrar sus velocidades $\vec{v}$ y coordenadas $\vec{r}$ en el siguiente instante $t = t_{next}$, suponiendo que su interacci\'on sea solamente gravitatoria. Si $\vec{r}_{i}$ y $m_{i}$ es la coordenada y masa para cada part\'icula, entones las ecuaciones de movimiento son
\begin{equation}
\frac{d^{2}\vec{r}_{i}}{d t^{2}}=
-G \sum_{j=1, i \not= j}^{N} \frac{m_{j}(\vec{r}_{i}-\vec{r}_{j})}{|\vec{r}_{i}-\vec{r}_{j}|^{3}}, \label{eqn 42}
\end{equation}
%donde $G$ es la consante de gravitaci\'on universal.
\end{frame}
%%%%%%%%%%%%%%%%%%%%%%%%%%%%%%%%%%%%%%%%%%%%%%%%%%%%%%%%%%%%%%%%%%%%%%%%%%%%%%%%%%%%%%%%%%%%%%%%%%%%%%%%%%%%%%%%%%%%%%%%
\begin{frame}
Se introduce un suavizante de fuerza: la fuerza se hace más débil (se suaviza) en distancias pequeñas para evitar aceleraciones grandes
\begin{equation}
\frac{d^{2}\vec{r}_{i}}{d t^{2}}=
-G \sum_{j=1, i \not= j}^{N} \frac{m_{j}(\vec{r}_{i}-\vec{r}_{j})}{(\Delta\vec{r}_{ij}^{2} + \epsilon^{2})^{3/2}},\label{eqn 43}
\end{equation} 
donde se ha definido $\Delta\vec{r}_{ij}^{2} = |\vec{r}_{i} - \vec{r}_{j}|$ y $\epsilon$ es el par\'ametro de suavizaci\'on o ``softening length''. Al considerar el modelo FLRW, la din\'amica de part\'iculas se describe con el Hamiltoniano
\begin{equation}
  H 
  = \sum_{i=1}^{N} \frac{\vec{p_{i}}^{2}}{2 m_{i} a^{2}(t)} 
  +
  \frac{1}{2} \sum_{i\not=1,i\not=j}^{N}\sum_{j=1,j\not=i}^{N}
  \frac{m_{i}m_{j}\Phi(\vec{x_{i}}-\vec{x_{j}})}{a(t)},\label{eqn 44}
\end{equation}
%donde $\vec{p_{k}}$ y $\vec{x_{k}}$ son los vectores de momento y posici\'on en el sistema de coordenadas com\'oviles y $a$ es el factor de escala.
\end{frame}
%%%%%%%%%%%%%%%%%%%%%%%%%%%%%%%%%%%%%%%%%%%%%%%%%%%%%%%%%%%%%%%%%%%%%%%%%%%%%%%%%%%%%%%%%%%%%%%%%%%%%%%%%%%%%%%%%%%%%%%%
\begin{frame}
El c\'alculo de fuerzas num\'erico se hace mediante un algoritmo tipo \'arbol o \textit{tree}. Este algoritmo divide en celdas el espacio de integraci\'on y c\'alculo de interacciones de manera jer\'arquica. Se calcula el potencial de interacci\'on entre part\'iculas y luego se aplica una Transformada de Fourier inversa para encontrar la fuerza de interacci\'on y viceversa. 
\begin{figure}
\centering
  \includegraphics[width=0.6\textwidth]{./Figuras/BHAlgorithm}
  \caption{\footnotesize{Una simulación de $100$-cuerpos utilizando el algoritmo tipo \'arbol.}}
  \label{fig 2.1}
\end{figure}
\end{frame}
%%%%%%%%%%%%%%%%%%%%%%%%%%%%%%%%%%%%%%%%%%%%%%%%%%%%%%%%%%%%%%%%%%%%%%%%%%%%%%%%%%%%%%%%%%%%%%%%%%%%%%%%%%%%%%%%%%%%%%%%
\begin{frame}
\frametitle{El C\'odigo}
GADGET (GAlaxies with Dark matter and Gas intEracT) es un c\'odigo de uso libre que utiliza la teor\'ia de $N$-cuerpos y SPH para el c\'alculo de simulaciones cosmol\'ogicas. Se utiliza una modificaci\'on del mismo: Axion-GADGET que se sirve del modelo de campo escalar con bos\'on ultraligero ($m \sim 10^{-22}$eV) y que tiene las ecuaciones de movimiento descritas en la secci\'on de campo escalar.
\end{frame}
%%%%%%%%%%%%%%%%%%%%%%%%%%%%%%%%%%%%%%%%%%%%%%%%%%%%%%%%%%%%%%%%%%%%%%%%%%%%%%%%%%%%%%%%%%%%%%%%%%%%%%%%%%%%%%%%%%%%%%%%
\begin{frame}
La naturaleza de FDM se describe mediante las ecuaciones de Schr\"odinger-Poisson
\begin{equation}
i\hbar \frac{d \Psi}{dt} 
=
-\frac{\hbar^{2}}{2m_{\chi}} \vec{\nabla}^{2}\Psi + m_{\chi}V\Psi,\label{eqn 3.1}
\end{equation}
y
\begin{equation}
\vec{\nabla}^{2}V = 4\pi G m_{\chi}|\Psi|^{2},\label{eqn 3.2}
\end{equation}
la funci\'on de onda se escribe como
\begin{equation}
\Psi = \sqrt{\frac{\rho}{m_{\chi}}}\exp(\frac{iS}{\hbar})\label{eqn 3.3}
\end{equation}
en t\'erminos de la densidad de n\'umero $\frac{\rho}{m_{\chi}}$, se define igualmente el momento lineal de la materia oscura como
\begin{equation}
\vec{\nabla}S = m_{\chi}\vec{v}.\label{eqn 3.4}
\end{equation}
\end{frame}

\begin{frame}
Con esta definici\'on, al introducir la función de onda $\Psi$ en la ecuaci\'on (\ref{eqn 3.1}) y resolviendo las ecuaciones de Schrödinger-Poisson, se obtiene la ecuación de continuidad
\begin{equation}
\frac{d\rho}{dt} + \vec{\nabla}\cdot(\rho\vec{v}) = 0,\label{eqn 3.5}
\end{equation}
y la ecuaci\'on de conservaci\'on de momento
\begin{equation}
\frac{d\vec{v}}{dt} + (\vec{v}\cdot\vec{\nabla})\vec{v} 
=
-\vec{\nabla}(Q + V).\label{eqn 3.6}
\end{equation}
Se obtiene el potencial cu\'antico $Q$
\begin{equation}
Q 
=
-\frac{\hbar^{2}}{2m^{2}_{\chi}}\frac{\vec{\nabla}^{2}\sqrt{\rho}}{\sqrt{\rho}}.\label{eqn 3.7}
\end{equation}
con la diferencia de que en la simlaci\'on, la interacci\'on entre part\'iculas no es relativista, se utiliza el operador Laplaciano $\nabla$ en lugar del D'Alambertiano $\Box$.
\end{frame}
%%%%%%%%%%%%%%%%%%%%%%%%%%%%%%%%%%%%%%%%%%%%%%%%%%%%%%%%%%%%%%%%%%%%%%%%%%%%%%%%%%%%%%%%%%%%%%%%%%%%%%%%%%%%%%%%%%%%%%%%%
\begin{frame}
El efecto de la presi\'on cu\'antica, se describe con el Hamiltoniano sin el t\'ermino de gravedad
\begin{equation}
H = \int \frac{\hbar^{2}}{2 m_{\chi}} |\vec{\nabla}\Psi|^{2}d^{3}x
  = \int \frac{\rho}{2} |\vec{v}|^{2}d^{3}x + \int \frac{\hbar^{2}}{2 m_{\chi}} (\vec{\nabla}\sqrt{\rho})^{2}d^{3}x.\label{eqn 3.8}
\end{equation}
La energía cin\'etica en forma discreta, con \'indice $j$ para identificar a cada part\'icula se escribe como
\begin{equation}
T = \int \frac{\rho}{2} |\vec{v}|^{2}d^{3}x = \sum_{j} \frac{1}{2} m_{j} \left(\frac{d q_{j}}{dt}\right)^{2}, \label{eqn 3.9}
\end{equation}
la energ\'ia potencial ser\'a
\begin{equation}
K_{p} = \int \frac{\hbar^{2}}{2 m_{\chi}} (\vec{\nabla}\sqrt{\rho})^{2}d^{3}x. \label{eqn 3.10}
\end{equation}
%Note que el término de la energía potencial $K_{p}$ no está discretizado aún. Esto se realizará más adelante.
Y el Lagrangiano
\begin{equation}
L = T - K_{p}
  =
\sum_{j} \frac{1}{2} m_{j} \left(\frac{d q_{j}}{dt}\right)^{2}
-
 \int \frac{\hbar^{2}}{2 m_{\chi}} (\vec{\nabla}\sqrt{\rho})^{2}d^{3}x,\label{eqn 3.11}  
\end{equation}
\end{frame}
%%%%%%%%%%%%%%%%%%%%%%%%%%%%%%%%%%%%%%%%%%%%%%%%%%%%%%%%%%%%%%%%%%%%%%%%%%%%%%%%%%%%%%%%%%%%%%%%%%%%%%%%%%%%%%%%%%%%%%%%
\begin{frame}
Las ecuaciones de Euler-Lagrange tendr\'an la forma
\begin{equation}
\frac{d}{dt}\frac{\partial L}{\partial \dot{q}_{j}} - \frac{\partial L }{\partial q_{j}} = 0 
\Rightarrow 
m_{j} \ddot{q}_{j} = - \frac{\partial K_{p}}{\partial q_{j}}.\label{eqn 3.12}
\end{equation}
Para una interacci\'on part\'icula-part\'icula, la densidad de n\'umero para cada part\'icula individual se describe con funciones delta
\begin{equation}
\rho(\vec{r})
=
\sum_{i}m_{i}\delta(\vec{r}-\vec{r}_{i}).\label{eqn 3.13}
\end{equation}
donde
\begin{equation}
\delta (\vec{r}-\vec{r}_{i}) = 
\frac{2\sqrt{2}}{\lambda ^{3} \pi ^{3/2}} \exp \left(-\frac{2|\vec{r}-\vec{r}_{i}|^{2}}{\lambda ^{2}}\right).\label{eqn 3.14}
\end{equation}

\end{frame}
%%%%%%%%%%%%%%%%%%%%%%%%%%%%%%%%%%%%%%%%%%%%%%%%%%%%%%%%%%%%%%%%%%%%%%%%%%%%%%%%%%%%%%%%%%%%%%%%%%%%%%%%%%%%%%%%%%%%%%%%
\begin{frame}
As\'i, el t\'ermino $(\vec{\nabla}\sqrt{\rho})^{2}$ se escribir\'a como
\begin{equation}
\begin{array}{ll}
\left[\vec{\nabla}\sqrt{\rho (\vec{r})}\right]^{2} &=
\frac{1}{4\rho(\vec{r})}\left[\sum_{i} m_{i}\vec{\nabla}\delta(\vec{r}-\vec{r}_{i})\right]^{2}, \\\\\\ 
&=
\frac{1}{4\rho(\vec{r})} 
\left[\sum_{i}
m_{i}\delta(\vec{r}-\vec{r}_{i})(-\frac{4}{\lambda^{2}})(\vec{r}-\vec{r}_{i})\right]^{2}, \\\\\\
&=
\frac{4}{\lambda^{4}\rho(\vec{r})}
\left[
\sum_{i} m_{i}\delta(\vec{r}-\vec{r}_{i})(\vec{r}-\vec{r}_{i})
\right]^{2}. \label{eqn 3.15}

\end{array}
\end{equation}
y la densidad 
\begin{equation}
\rho(\vec{r})
=
\sum_{j}\sum_{i}m_{i}\delta(\vec{r}-\vec{r}_{j}),\label{eqn 3.16}
\end{equation} 
con $j$ denotandocada agrupaci\'on de c\'umulos. Luego
\begin{equation}
\left[\vec{\nabla}\sqrt{\rho (\vec{r})}\right]^{2} \simeq 
\frac{4}{\lambda^{4}}
\left[
\sum_{i}m_{i}\delta(\vec{r}-\vec{r}_{j})(\vec{r}-\vec{r}_{j})
\right]^{2}
\left[
\sum_{j}m_{j}\delta(\vec{r}-\vec{r}_{j})
\right]^{-1}. \label{eqn 3.17}
\end{equation}

\end{frame}
%%%%%%%%%%%%%%%%%%%%%%%%%%%%%%%%%%%%%%%%%%%%%%%%%%%%%%%%%%%%%%%%%%%%%%%%%%%%%%%%%%%%%%%%%%%%%%%%%%%%%%%%%%%%%%%%%%%%%%%%%
\begin{frame}
Integrando la ecuaci\'on (\ref{eqn 3.10}) se tiene
\begin{equation}
\int\left[\vec{\nabla}\sqrt{\rho (\vec{r})}\right]^{2}  \simeq 
\int \footnotesize{\frac{4 d^{3}x}{\lambda^{4}}
\left[
\sum_{j} m_{j}\delta(\vec{r}-\vec{r}_{j})(\vec{r}-\vec{r}_{j})
\right]^{2}
\left[
\sum_{j}m_{j}\delta(\vec{r}-\vec{r}_{j})
\right]^{-1}}\label{eqn 3.18}
\end{equation}
\begin{equation}
\simeq
\footnotesize{4\lambda ^{-4} 
\sum_{j}m_{j}\delta(\vec{r}-\vec{r}_{j})(\vec{r}-\vec{r}_{j})^{2}\Delta V_{j}B_{j},}\label{eqn 3.19}
\end{equation}
\begin{equation}
\simeq
\footnotesize{4\lambda^{-4}
\sum_{j}m_{j}
\frac{\Delta V_{j}B_{j}}{\lambda^{3}\pi^{3/2}}
\exp\left[-\frac{(\vec{r}-\vec{r}_{j})^{2}}{\lambda^{2}}\right]
(\vec{r}-\vec{r}_{j})^{2},} \label{eqn 3.20}
\end{equation}
donde $V_{j}$ y $B_{j}$ son el volumen efectivo y el factor de correcci\'on de la $j$-\'esima partícula. 
\end{frame}
%%%%%%%%%%%%%%%%%%%%%%%%%%%%%%%%%%%%%%%%%%%%%%%%%%%%%%%%%%%%%%%%%%%%%%%%%%%%%%%%%%%%%%%%%%%%%%%%%%%%%%%%%%%%%%%%%%%%%%%%
\begin{frame}
Finalmente, la ecuaci\'on (\ref{eqn 3.10}) se reacomoda
\begin{equation}
\footnotesize{\sum_{j} \frac{\partial K_{p}}{\partial q_{j}}
=
\frac{4\hbar^{2}}{m_{\chi}^{2}\lambda^{4}}
\sum_{j}m_{j}\Delta V_{j}B_{j}
\exp\left[
-\frac{2|\vec{r}-\vec{r}_{j}|^{2}}{\lambda^{2}}\right]
(1-\frac{2|\vec{r}-\vec{r}_{j}|^{2}}{\lambda^{2}})
(\vec{r}-\vec{r}_{j})}\label{eqn 3.21}
\end{equation}
al igual que la ecuaci\'on de movimiento (\ref{eqn 3.12})
\begin{equation}
\footnotesize{\sum_{j}m_{j}\ddot{q}_{j}
=
\frac{4\hbar^{2}}{m_{\chi}^{2}\lambda^{4}}
m_{j}\Delta V_{j}B_{j}
\exp\left[
-\frac{2|\vec{r}-\vec{r}_{j}|^{2}}{\lambda^{2}}\right]
(1-\frac{2|\vec{r}-\vec{r}_{j}|^{2}}{\lambda^{2}})
(\vec{r}-\vec{r}_{j}).}\label{eqn 3.22}
\end{equation}
Sustituyendo $q$ con $\vec{r}$, la aceleraci\'on adicional de la presi\'on cu\'antica en la simulaci\'on se describe como
\begin{equation}
\footnotesize{\ddot{\vec{r}}
=
\frac{4M\hbar^{2}}{M_{0}m_{\chi}^{2}\lambda^{4}}
\sum_{j}B_{j}
\exp\left[
-\frac{2|\vec{r}-\vec{r}_{j}|^{2}}{\lambda^{2}}\right]
\left(1-\frac{2|\vec{r}-\vec{r}_{j}|^{2}}{\lambda^{2}}\right)
(\vec{r}_{j}-\vec{r}).}\label{eqn 3.23}
\end{equation}
La presi\'on cu\'antica es una interacci\'on de rango corto y es atractiva (repulsiva) si la distancia entre part\'iculas es mayor (menor) a $\lambda/\sqrt{2}$.

\end{frame}
%%%%%%%%%%%%%%%%%%%%%%%%%%%%%%%%%%%%%%%%%%%%%%%%%%%%%%%%%%%%%%%%%%%%%%%%%%%%%%%%%%%%%%%%%%%%%%%%%%%%%%%%%%%%%%%%%%%%%%%%
%%%%%%%%%%%%%%%%%%%%%%%%%%%%%%%%%%%%%%%%%%%%%%%%%%%%%%%%%%%%%%%%%%%%%%%%%%%%%%%%%%%%%%%%%%%%%%%%%%%%%%%%%%%%%%%%%%%%%%%%

%%%%%%%%%%%%%%%%%%%%%%%%%%%%%%%%%%%%%%%%%%%%%%%%%%%%%%%%%%%%%%%%%%%%%%%%%%%%%%%%%%%%%%%%%%%%%%%%%%%%%%%%%%%%%%%%%%%%%%%%%
%%%%%%%%%%%%%%%%%%%%%%%%%%%%%%%%%%%%%%%%%%%%%%%%%%%%%%%%%%%%%%%%%%%%%%%%%%%%%%%%%%%%%%%%%%%%%%%%%%%%%%%%%%%%%%%%%%%%%%%%
\begin{frame}
\frametitle{Resultados}
\begin{table}[htb]
\centering
\begin{tabular}{|l|l|l|l|}
\hline
& \multicolumn{3}{c|}{Condiciones Iniciales} \\
\cline{2-4}
& Descripci\'on & S\'imbolo & Valor\\
\hline \hline
\multirow{3}{2cm}{Densidades \\ ($z=0$)}
& Materia oscura & $\Omega$ & 0.268\\ \cline{2-4}
& Constante cosmol\'ogica & $\Omega_{\Lambda}$ & 0.683\\ \cline{2-4}
& Materia Bari\'onica &  $\Omega_{b}$ & 0.049\\ \cline{1-4}
\hline
Tama\~no de simulaci\'on & Boxsize & $L$ & 5 Mpc\\ \cline{1-4}
\hline
N\'umero de part\'iculas & $N$ & $2^{21}\sim$ &2M\\ \cline{1-4}
\multirow{3}{1cm}{Tiempo} & Redshift inicial & $z_{i}$ & 15\\ \cline{2-4}
& Redshift final & $z_{f}$ & 0\\ \cline{1-4}
\hline
\multirow{3}{3cm}{Otros par\'ametros} & Par\'ametro de Hubble & $h$  & 0.7\\         \cline{2-4}
& espectro de potencias & $\sigma_{8}$ & 0.8\\ \cline{1-4}
%&Suavizamiento & $\epsilon$ & 1.81 kpc\\ \cline{1-4}
\end{tabular}
\caption{Condiciones creadas con N-GenIC.}
%\label{tabla:final}
\end{table}
\end{frame}
%%%%%%%%%%%%%%%%%%%%%%%%%%%%%%%%%%%%%%%%%%%%%%%%%%%%%%%%%%%%%%%%%%%%%%%%%%%%%%%%%%%%%%%%%%%%%%%%%%%%%%%%%%%%%%%%%%%%%%%%%%%%%
\begin{frame}
\begin{table}[htb]
\centering
\begin{tabular}{|l|l|l|l|}
\hline
& \multicolumn{3}{c|}{Par\'ametros F\'isicos} \\
\cline{2-4}
& Descripci\'on & Cantidad & Unidad\\
\hline \hline
\multirow{3}{2cm}{Sistema de Unidades}
& Longitud (cm) &  $3.08\times 10^{21}$  & 1 kpc\\ \cline{2-4}
& Masa (g) & $1.989\times 10^{43}$ & $10^{10}$ $M_{\odot}$\\ \cline{2-4}
& Velocidad (cm/s) & $10^{5}$  &1 km/s\\ \cline{1-4}
\multirow{3}{1cm}{Suavizados} & $\Lambda CDM$ & $\epsilon$ & [0.5,1.81] kpc\\ \cline{2-4}
& SFDM & $\epsilon$ & [0.5,1.81] kpc\\ \cline{1-4}
\multirow{3}{2cm}{SFDM} & FdmAxionMass & $M_{\chi}$  & $10^{-22}$ eV \\ \cline{2-4}
& FdmKernelLambda & $\lambda$ & 1.14213562 kpc \\ \cline{1-4}
\end{tabular}
\caption{Cantidades adicionales para las simulaciones.}
%\label{tabla:final}
\end{table}
\end{frame}
%%%%%%%%%%%%%%%%%%%%%%%%%%%%%%%%%%%%%%%%%%%%%%%%%%%%%%%%%%%%%%%%%%%%%%%%%%%%%%%%%%%%%%%%%%%%%%%%%%%%%%%%%%%%%%
\begin{frame}
\begin{figure}
\centering
\includegraphics[width=0.8\textwidth]{./Figuras/xy_00081_image}
\end{figure}
\end{frame}
%%%%%%%%%%%%%%%%%%%%%%%%%%%%%%%%%%%%%%%%%%%%%%%%%%%%%%%%%%%%%%%%%%%%%%%%%%%%%%%%%%%%%%%%%%%%%%%%%%%%%%%%%%%%%%%%%%%%%%%%%%%%%%%%%%%%%%%%%%%%%%%%%%%%%
\begin{frame}
\begin{figure}
\includegraphics[width=0.8\textwidth]{./Figuras/masses_LCDM2M}
\caption{\footnotesize{Perfil de masas de halos de materia oscura, simulaci\'on de $\Lambda$CDM}}
\end{figure}
\end{frame}
%%%%%%%%%%%%%%%%%%%%%%%%%%%%%%%%%%%%%%%%%%%%%%%%%%%%%%%%%%%%%%%%%%%%%%%%%%%%%%%%%%%%%%%%%%%%%%%%%%%%%%%%%%%%%%%%%%%%%%%%%%%%%%%%
\begin{frame}
\begin{figure}[htpb]
\centering
\subfigure[]{\includegraphics[width=0.3\textwidth]{./Figuras/movie_xy_00000}}
\subfigure[]{\includegraphics[width=0.3\textwidth]{./Figuras/movie_xy_00030}}
\subfigure[]{\includegraphics[width=0.3\textwidth]{./Figuras/movie_xy_00101}}
\caption{\footnotesize{Evoluci\'on de dos galaxias espirales colisionando para formar una sola. En la figura las partículas azules representan el halo de materia oscura, las partículas rojas representan el disco estelar.}} \label{fig 3.3}
\end{figure}
\end{frame}
\end{document}